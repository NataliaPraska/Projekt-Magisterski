\chapter{Badania końcowe i wnioski projektowe}

\section{Cel eksperymentu}
 Celem przeprowadzenia badania było empiryczne porównanie wpływu dwóch różnych podejść do projektowania interfejsu użytkownika oraz wybranych mechanizmów interakcji w wirtualnym środowisku. Analiza obejmowała typ interfejsu, sposób przemieszczania się oraz intensywność informacji zwrotnej, czyli liczbę kanałów komunikujących reakcje systemu na wykonane zadania. Miały również ocenić poziom immersji i poczucia obecności użytkownika podczas wykonywania czterech zadań. Porównanie obu wariantów przeprowadzono w kontrolowanych warunkach, zachowując jednocześnie identyczną strukturę zadań, układu i celów do wykonania. Rożnicę pomiędzy wariantem 1 i 2 ograniczono do badanych zmiennych, aby następnie zestawić zachowania użytkowników i ocenić wpływ decyzji projektowych na przebieg rozgrywki 
 
\section{Metodologia badań}

 Zakres badanych obszarów został wyselekcjonowany na podstawie wcześniejszych wyników analizy heurystycznej oraz badania ankietowego. Do badań włączono te aspekty interakcji, które okazały się problematyczne lub niejednoznaczne w obu etapach analizy. Jednym z najczęściej wskazywanych problemów w doświadczeniach VR były mechanizmy lokomocji. Informacja zwrotna, mimo że nie była wskazywana przez użytkowników jako obszar krytyczny, została uwzględniona w badaniu ze względu na wyniki analizy heurystycznej. W większości analizowanych gier VR była ona ograniczona lub pomijana, dlatego zdecydowano się na jej uwzględnienie w badaniu końcowym w celu oceny wpływu intensywności informacji zwrotnej na jakość rozgrywki i przebieg interakcji.
 
 Z zakresu badań świadomie zrezygnowano również z mechaniki ekwipunku, pomimo problemów wykrytych w pierwszym etapie analizy. Decyzja ta wynikała z faktu, że zidentyfikowane trudności dotyczyły wyłącznie jednego typu rozwiązania, tj. ekwipunku realizowanego w formie panelu osadzonego w przestrzeni. Pozostałe warianty uzyskały dobre oceny użyteczności i były wskazywane w badaniu ankietowym jako rozwiązania udane. Uznano zatem, że uwzględnienie mechaniki ekwipunku nie wniesie istotnej wartości do dalszych badań, a zakres pracy zawężono do obszarów generujących największe i najczęściej występujące problemy użytkowe. Dodatkowo obszar ten pokrywał się częściowo funkcjonalnie z interfejsem panelowym, który został już objęty zakresem badania.
W celu wyeliminowania efektu uczenia się badanie zostało zaprojektowane zgodnie z modelem międzygrupowym i przeprowadzone stacjonarnie. Uczestnicy zostali podzieleni na dwie niezależne grupy badawcze, a każda z nich testowała wyłącznie jeden wariant interfejsu. Wybór projektu międzygrupowego został podyktowany obawą, że schemat wewnątrzgrupowy mógłby prowadzić do zapamiętania schematu zadania przez użytkowników i zakłócenia wyników. W badaniu wzięło udział 10 dorosłych osób a rekrutacja odbywała się wśród znajomych i rodziny. Wybrano 6 osób, które miały niewielką styczność z projektowanym środowiskiem, nieprzekraczającą około 30 minut wcześniejszego kontaktu, oraz 4 osoby posiadające minimum 10h doświadczenia z urządzeniami VR. Przydzielenie osób doświadczonych i początkujących odbyło się losowo, tak aby powstałe grupy były równo zróżnicowane pod względem poziomu zaawansowania z technologią VR. Decyzja o takim podziale wynikała z wyników ankiet, w których to duża część ankietowanych wskazała na jednokrotne korzystanie z tej technologii. Każda z nich zawierała 3 użytkowników początkujących i 2 zaawansowanych. Zapewniło to kontrole nad możliwym rożnym poziomem doświadczenia osób badanych i pozwoliło na sprawdzenie, czy wcześniejsze doświadczenie wpłynie na odbiór danego wariantu. 

	\begin{figure}[!htb]
    \centering
    \includegraphics[width=1
    \textwidth]{images/chapter6/1.png}
    \caption{Schemat przebiegu eksperymentu międzygrupowego w VR., (2026); źródło: własne}
    \end{figure}


Zaprojektowane zadania nie miały określonego limitu czasu na zadanie aby umożliwić naturalne zachowanie i swobodną eksploracje. W przypadku trudności badanie nie było przerywane, ale dopuszczono przerwy spowodowane złym samopoczuciem. Aby nie wpływać na przebieg badania podczas wykonywania zadań nie udzielano żadnych podpowiedzi. Dane ilościowe były zapisywane automatycznie przez system. Badanie opierało się wyłącznie na metrykach zapisywanych przez system i notatkach obserwacyjnych.

Zastosowanie tej metody miało na celu weryfikację, w jakim stopniu dany prototyp wspiera użytkownika w wykonywaniu założonych czynności \cite{Interakcja}. Dodatkową zaletą obserwacji była możliwość identyfikacji przyczyn zróżnicowanych wyników między poszczególnymi scenariuszami, co pozwoliło określić, które elementy środowiska sprzyjają efektywnej interakcji, a które wymagają dalszych modyfikacji i optymalizacji.

\section{Wyniki i interpretacja}
Czas wykonania zadań rejestrowano automatycznie przez system w sekundach, z dokładnością do dwóch miejsc po przecinku. Jest to podstawowa metryka użyteczności, opisująca czas potrzebny użytkownikowi do ukończenia zdefiniowanego zadania \cite{Pomiary}. Po badaniu dane eksportowano do pliku \texttt{.txt}, natomiast obserwacje jakościowe zapisywano odręcznie na tablecie w trakcie trwania eksperymentu. W tabelach zestawiono czasy wykonania zadań oraz liczbę niepowodzeń. Jako niepowodzenie (N) w zadaniu 1 uznawano upuszczenie przenoszonego przedmiotu. W zadaniu 2 niepowodzenie stanowiło wprowadzenie błędnej sekwencji. W zadaniach 3 i 4 za niepowodzenie przyjmowano upadek użytkownika (np. spadnięcie z przeszkody lub platformy), skutkujący koniecznością cofnięcia się i ponownej próby.

\begin{table}[htbp]
\centering
\caption{Wyniki wykonania zadań w scenariuszu A (czas oraz liczba niepowodzeń $N$)}
\label{tab:scenariuszA_wyniki}
\renewcommand{\arraystretch}{1.2}
\setlength{\tabcolsep}{3pt}
\footnotesize

\resizebox{\textwidth}{!}{%
\begin{tabular}{|c|c|c|c|c|c|c|c|c|c|c|}
\hline
\textbf{Użytkownik} & \textbf{Poziom} &
\multicolumn{2}{c|}{\textbf{Zadanie 1}} &
\multicolumn{2}{c|}{\textbf{Zadanie 2}} &
\multicolumn{2}{c|}{\textbf{Zadanie 3}} &
\multicolumn{2}{c|}{\textbf{Zadanie 4}} &
\textbf{Czas ukończenia} \\
\cline{3-10}
 &  &
\textbf{Czas [s]} & \textbf{$N$} &
\textbf{Czas [s]} & \textbf{$N$} &
\textbf{Czas [s]} & \textbf{$N$} &
\textbf{Czas [s]} & \textbf{$N$} &
\textbf{scenariusza [s]} \\
\hline
U1 & Początkujący & 104,54 & 0 & 98,16  & 2 & 84,75  & 1 & 78,52  & 0 & 365,97 \\
U2 & Początkujący & 96,85  & 0 & 93,23  & 2 & 118,09 & 4 & 106,98 & 1 & 411,15 \\
U3 & Początkujący & 110,98 & 1 & 123,02 & 4 & 99,11  & 2 & 69,75  & 0 & 402,86 \\
U4 & Zaawansowany & 78,31  & 0 & 78,32  & 0 & 27,98  & 0 & 55,43  & 0 & 240,04 \\
U5 & Zaawansowany & 73,73  & 0 & 65,25  & 0 & 37,77  & 0 & 59,01  & 0 & 235,76 \\
\hline
\multicolumn{2}{|l|}{\textbf{Średnia czasu / osoby z niepowodzeniem (Początkujący)}} &
104,12 & 1/3 & 104,80 & 3/3 & 100,65 & 3/3 & 85,08 & 1/3 & 390,99 \\
\hline
\multicolumn{2}{|l|}{\textbf{Średnia czasu / osoby z niepowodzeniem (Zaawansowani)}} &
76,02 & 0/2 & 71,79 & 0/2 & 32,88 & 0/2 & 57,22 & 0/2 & 237,90 \\
\hline
\end{tabular}%
}

\vspace{0.5em}
\footnotesize
\textbf{Uwaga:} $N$ oznacza liczbę niepowodzeń w zadaniu. W wierszach podsumowania podano średni czas oraz liczbę osób z co najmniej jednym niepowodzeniem w formacie $k/n$.
\end{table}

Na podstawie tabeli \ref{tab:scenariuszA_wyniki} można zauważyć wyraźny wpływ doświadczenia na przebieg wykonywania zadań. Osoby zaawansowane (U4–U5) zrealizowały scenariusz znacznie szybciej (średni czas ukończenia 237,90~s) niż uczestnicy początkujący (390,99~s). W grupie zaawansowanej nie odnotowano żadnych niepowodzeń, co wskazuje na większą pewność i płynność podczas wykonywania interakcji.

W grupie początkującej największym wyzwaniem okazało się zadanie 3. Każdy z uczestników odnotował co najmniej jedno niepowodzenie (3/3), a u części badanych liczba niepowodzeń była wysoka (do 4), co przekładało się na dłuższy czas realizacji. Zadanie 2 generowało podobne trudności, natomiast zadania 1 i 4 okazały się bardziej przystępne.

Różnice między grupami były najbardziej widoczne w zadaniu 3, gdzie średni czas w grupie początkującej wyniósł 100,65~s, a w grupie zaawansowanej 32,88~s (ponad trzykrotnie krócej). Dodatkowo wyniki grupy początkującej charakteryzowały się większym rozrzutem czasu ukończenia scenariusza (365,97--415,15~s) w porównaniu do grupy zaawansowanej (235,76--240,04~s).


\begin{table}[htbp]
\centering
\caption{Wyniki wykonania zadań w scenariuszu B (czas oraz liczba niepowodzeń $N$)}
\label{tab:scenariuszB_wyniki}
\renewcommand{\arraystretch}{1.2}
\setlength{\tabcolsep}{3pt}
\footnotesize

\resizebox{\textwidth}{!}{%
\begin{tabular}{|c|c|c|c|c|c|c|c|c|c|c|}
\hline
\textbf{Użytkownik} & \textbf{Poziom} &
\multicolumn{2}{c|}{\textbf{Zadanie 1}} &
\multicolumn{2}{c|}{\textbf{Zadanie 2}} &
\multicolumn{2}{c|}{\textbf{Zadanie 3}} &
\multicolumn{2}{c|}{\textbf{Zadanie 4}} &
\textbf{Czas ukończenia} \\
\cline{3-10}
 &  &
\textbf{Czas [s]} & \textbf{$N$} &
\textbf{Czas [s]} & \textbf{$N$} &
\textbf{Czas [s]} & \textbf{$N$} &
\textbf{Czas [s]} & \textbf{$N$} &
\textbf{scenariusza [s]} \\
\hline
U6  & Początkujący & 148,26 & 1 & 88,99 & 1 & 74,55 & 1 & 139,97 & 1 & 451,77 \\
U7  & Początkujący & 124,65 & 0 & 79,95 & 1 & 67,91 & 0 & 220,28 & 3 & 492,79 \\
U8  & Początkujący & 131,93 & 1 & 82,91 & 1 & 68,37 & 0 & 90,91  & 0 & 374,12 \\
U9  & Zaawansowany & 87,17  & 0 & 52,02 & 0 & 20,93 & 0 & 54,21  & 0 & 214,33 \\
U10 & Zaawansowany & 82,58  & 0 & 51,25 & 0 & 22,62 & 0 & 56,66  & 0 & 213,11 \\
\hline
\multicolumn{2}{|l|}{\textbf{Średnia czasu / osoby z niepowodzeniem (Początkujący)}} &
134,95 & 2/3 & 83,95 & 1/3 & 70,28 & 1/3 & 150,39 & 2/3 & 439,56 \\
\hline
\multicolumn{2}{|l|}{\textbf{Średnia czasu / osoby z niepowodzeniem (Zaawansowani)}} &
84,88 & 0/2 & 51,64 & 0/2 & 21,78 & 0/2 & 55,44 & 0/2 & 213,72 \\
\hline
\end{tabular}%
}

\vspace{0.5em}
\footnotesize
\textbf{Uwaga:} $N$ oznacza liczbę niepowodzeń w zadaniu. W wierszach podsumowania podano średni czas oraz liczbę osób z co najmniej jednym niepowodzeniem w formacie $k/n$.
\end{table}


Na podstawie danych zestawionych w tabeli \ref{tab:scenariuszB_wyniki} można zaobserwować wyraźny wpływ poziomu doświadczenia użytkowników na przebieg realizacji zadań. Uczestnicy zaawansowani (U9–U10) ukończyli scenariusz istotnie szybciej (średni czas 213{,}72~s) niż uczestnicy początkujący (439{,}56~s). W grupie zaawansowanej nie odnotowano żadnych niepowodzeń w realizacji poszczególnych zadań (0/2), co wskazuje na stabilne, płynne i pewne wykonywanie interakcji.

W grupie początkującej największe trudności zaobserwowano w zadaniu 4, w którym niepowodzenia wystąpiły u 2/3 uczestników, przy czym w jednym przypadku odnotowano ich wysoką liczbę (do 3), co wiązało się z wyraźnym wydłużeniem czasu realizacji tego zadania. Podwyższony poziom trudności ujawnił się również w zadaniu 1 (niepowodzenia u 2/3 osób). W zadaniach 2 i 3 niepowodzenia występowały rzadziej (odpowiednio 1/3 oraz 1/3 badanych), a czasy realizacji charakteryzowały się większą jednorodnością.

Największe różnice pomiędzy grupami ujawniły się w zadaniu 4: średni czas wykonania w grupie początkującej wyniósł 150,39~s, podczas gdy w grupie zaawansowanej 55,44~s. Ponadto wyniki grupy początkującej cechowały się wyraźnie większą zmiennością czasu ukończenia całego scenariusza (374,12--492,79~s) w porównaniu z grupą zaawansowaną (213,11--214,33~s).


Obserwacje jakościowe







W scenariuszu A wskazówki potrzebne do rozwiązania zadania zostały celowo umieszczone w osobnym pomieszczeniu, aby sprawdzić, czy użytkownicy będą aktywnie poszukiwać informacji w otoczeniu. Uczestnicy testujący wariant z interfejsem panelowym często nie powiązali etapu wpisywania kombinacji z koniecznością wcześniejszego odnalezienia wskazówki w przestrzeni i rozpoczęli wpisywanie sekwencji metodą prób, co znajduje odzwierciedlenie w liczbie zarejestrowanych niepowodzeń. Uczestnik z największą liczbą prób ostatecznie trafił na poprawne hasło poprzez zgadywanie, natomiast dwie osoby z mniejszą liczbą niepowodzeń po pierwszych próbach zinterpretowały informację zwrotną jako sygnał, że wprowadzone dane są nieprawidłowe i że hasło powinno być gdzieś podane.

Warto zaznaczyć, że w tym wariancie informacja zwrotna była ograniczona i pojawiała się dopiero po wpisaniu całej sekwencji, co sprawiało, że uczestnicy nie otrzymywali sygnału już przy pierwszym popełnionym błędzie. Taki sposób udzielania feedbacku sprzyjał dalszemu podejmowaniu prób, ponieważ użytkownik nie miał natychmiastowej informacji, że wybrana strategia jest nieprawidłowa ani jednoznacznej sugestii, że rozwiązania należy szukać w innym obszarze.






Podczas badania uczestnicy mieli możliwość dzielenia się spontaniczną uwagą lub sugestią dotyczącą przetestowanego wariantu interfejsu. Odpowiedzi zapisywano w postaci krótkich notatek. Element ten nie stanowił formalnej części badania, lecz umożliwiał zebranie dodatkowych wskazówek projektowych, które były nieuchwytne dla automatycznych pomiarów ani obserwacji zachowań.
\section{Wnioski projektowe}




\section{Ograniczenia badań}

Przeprowadzone badanie wiąże sie z kilkoma ograniczeniami, które wpływają na zakres otrzymanych wniosków. Ograniczenia przeprowadzonego badania wynikają przede wszystkim z dostępnych zasobów. Były to świadome decyzje wpływające na zakres otrzymanych wniosków. Próba badawcza wynosiła 10 osób, co pozwalało na wychwycenie tendencji i różnic pomiędzy poszczególnymi wariantami. Nie daje jednak podstaw do wyciągania wniosków dotyczących całej populacji. Zastosowanie schematu międzygrupowego wyeliminowało efekt uczenia się, ale zwiększyło wpływ indywidualnych różnic miedzy użytkownikami. Pomimo zachowania równowagi składu grup, niewielka liczba uczestników uniemożliwiła przeprowadzenie dogłębnej analizy wpływu poziomu zaawansowania na odbiór interfejsu i interakcji. Badania zrealizowano w warunkach domowych, wykorzystując jedną konfigurację sprzętową. Dostępne były inne zestawy VR, jednak nie oferowały one funkcji, które mogłyby znacząco rozszerzyć zakres pomiarów (np. o pomiary fizjologiczne). Realizacja testów w środowisku domowym zapewniła większą elastyczność organizacyjną i umożliwiła dostosowanie terminów sesji do uczestników, jednak uzyskane wyniki odnoszą się do konkretnej konfiguracji testowej. Na zaawansowanym etapie realizacji pracy odnaleziono zagraniczne opracowanie o zbliżonej tematyce, przez co zawarte w nim wytyczne nie mogły zostać uwzględnione w projekcie ani w procedurze badania, co stanowi dodatkowe ograniczenie.

\section{Podsumowanie pracy}



