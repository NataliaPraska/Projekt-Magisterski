\chapter{Analiza istniejących rozwiązań}

\section{ Cel i zakres analizy}
Opis celu: identyfikacja sposobów realizacji kluczowych funkcji UI w grach VR.
Wymienienie analizowanych funkcji: menu pauzy, ekwipunek/inwentarz, tryb poruszania się, informacja zwrotna.
Wyjaśnienie, że analiza służy jako podstawa do dalszej weryfikacji ankietowej


\section{Metoda analizy: heurystyczna ocena wzorców projektowych}
Opis mojego dwuetapowego  podejścia:

Identyfikacja wzorców jak to się robi?
Ocena według heurystyk NN/g (2021) dostosowanych do VR.
– Lista zastosowanych heurystyk (bez szczegółów naruszeń):
H1. Visibility of System Status
H2. Match Between System and the Real World
H3. User Control and Freedom
H4. Consistency and Standards
H5. Error Prevention


\section{ Materiał badawczy (wybór gier VR)}
– Kryteria wyboru:

popularność
różnorodność gatunków
dostępność na komercyjnych platformach
obecność interfejsu w przestrzeni 3D

Lista gier:

\section{Identyfikacja wzorców projektowych}
\section{Ustalenie zestawu heurystyk do oceny}
\section{Ocena gier według heurystyk}
\section{Podsumowanie analizy i Wnioski}

