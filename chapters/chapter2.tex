\chapter{Podstawy teoretyczne i Tło technologiczne}

\section{Podstawy teoretyczne} 


\subsection{Definicje, pojęcia i klasyfikacje VR}
Aby uzasadnić potrzebe projektowania interfejsów i interakcji w grach VR w odmienny sposób niż w przypadku gier 2D czy 3D, należy najpierw zrozumieć czym jest tak naprawde VR. 
W niniejszej pracy przyjęto ujęcie rzeczywistości wirtualnej zaproponowane w książce The VR Book autorstwa Jasona Jeralda. Rzeczywistość wirtualna rozumiana jest jako w pełni sztuczne, cyfrowe otoczenie generowane komputerowo, które wywołuje u użytkownika poczucie przebywania w innym miejscu lub świecie oraz umożliwia mu bezpośrednie doświadczanie i interakcję w sposób zbliżony do kontaktu z rzeczywistym środowiskiem (Jerald, The VR Book). W odróznieniu od tradycyjnych gier komputerowych oraz aplikacji, technologia ta nie ogranicza się tylko do obserwowania obrazu na ekranie lecz zakłada pełne zaangażowanie użytkownika. 

Dzięki specjalistycznym urządzeniom, takim jak gogle czy kontrolery ruchu, możliwe staje się obserwowanie i oddziaływanie na generowane wirtualnie obiekty w czasie rzeczywistym. Rozwiązania z zakresu VR wykorzystują zaawansowane techniki renderowania grafiki, precyzyjnie śledzą ruch głowy i dłoni, a także uwzględniają dźwięk przestrzenny. Może angażować nie tylko słuch i wzrok ale również dotyk poprzez kontrolery haptyczne. Wirtualna rzeczywistość odziałowywuje na różne \textbf{zmysły}, aby maksymalnie zwiększyć zanurzenie użytkownika i osiągnięcie wrażenia, że świat wirtualny jest prawdziwy a użytkownik jest w nim naprawdę obecny. 

Do opisu doświadczeń w grach często stosowane jest pojęcie immersja oraz obecność. Słowa te są czesto mylnie stosowane jako synonimy choć odnoszą się do rożnych aspektów VR. Różnice między tymi pojęciami dość trafnie opisał Jerald w swojej książce book. Zaznacza on, że Imersja nawiązuje do cech VR, do tego w jak mocno oddziałowują na na zmysły użytkownika. Natomiast obecność odnosi się do subiektywnego wrażenia i odczucia użytkownika że znajduje się w świecie wirtualnym. W kontekscie gier VR te słowa mają szczególne znaczenie, są zależne od tego jak zaprojektowana jest cała gra. 

\begin{figure}[!htb]
    \centering
    \includegraphics[width=0.8\textwidth]{images/vr_example.jpg}
    \caption{Symulator medyczny służący do szkolenia lekarzy}
    \label{vr_example}
\end{figure}\textbf{}


\subsection{Ergonomia, percepcja i ograniczenia użytkownika}

W grach komputerowych czy aplikacjach rzadko zdarza się aby użytkownik odczuwał dyskomfort fizyczny. VR angażuje urzytkownika intensywniej niż tradycyjne rozgrywki na komputer i pozwala użytkownikowi na pełne zanurzenie się w rozgrywkę. Tworzenie treści do tego środowiska wymaga uwzględnienia ograniczeń percepcyjnych i fizjologicznych człowieka. Jednym z efektów braku takiego podejścia jest wystąpienie objawów choroby symulatorowej (VR sickness). Objawia się najczęściej mdłościami, zawrotami głowy czy ogólna dezorientacją. Według ksiązki 3D User Interfaces: Theory and Practice (book) (3D user Interfaces Theory and Practice) jest ona spowodowana niespójnością pomiędzy tym, co użytkownik widzi czyli ruch w świecie wirtualnym, a tym, co odczuwa jego ciało czyli brak fizycznego ruchu. Z tego wzgledu bardzo istotne jest tempo interakcji, ograniczenie gwałtwonych zmian perspektywy i widoku oraz przewidywalność zachowań systemu ponieważ to miedzy innymi one będą miały wpływ na długie i komfortowe korzystanie z urządzenia. 

\begin{figure}[!htb]
    \centering
    \includegraphics[width=0.8\textwidth]{images/VRSICK.png}
    \caption{VR sickness}
    \label{unity_engine_example}
\end{figure}
(3D user Interfaces Theory and Practice)


Zaangażowanie całego ciała w przypadku gier VR nakłada również pewne ograniczenia na projektantów. Jednym z nich jest to że niewłasciwe ułożenie obiektów na scenie lub wymuszenie utrzymywania dłoni w nienaturalnej pozycji przez dłuższy czas może prowadzić do szybkiego zmęczenia. Z tego powody należy uwzględnić naturalny zakres ruchu człowieka aby wszelkiego rodzaju interakcje mogły być wykonywane bez nadmiernego wysiłku fizycznego.  (book) jerald

Problematycznym może okazać się samo umieszczenie obiektów interaktywnych w środowisko wirtualnym. Pole widzenia w googlach jest ograniczone nie tylko ze względu na konstrukcje urządzeni ale również naszych indywidualnych cech anatomicznych jak rozstaw źrenic. Ma to istotny wpływ na odbiór interfejsu i rozmieszczenie informacji w przestrzeni. Jak wykazują badania  Sauer i współpracowników (2022),  article (Assessment of consumer VR). pole widzenia jakie deklarują producenci zestawów VR jest inne niż rzeczywistosci a jego zakres zależy fizjologi odbiorcy. Umieszczenie kluczowych elementów poza zakresem widzialnym będzie prowadził do ich przeoczenia lub wymuszenia zbędnych ruchów i w rezultacie zmęczenia lub irytacji. 
%Ten fragment cały ma informacje z dwoch ksiazek i jednego artykulu i nie wiem jak cytować  (Jerald, 2015; LaViola et al., 2017; Sauer et al., 2022).
Z tego względu należy stosować nie tylko teoretyczne założenia techniczne podane przez producenta ale również empiryczne ograniczenia odbiorców. 

 

\subsection{Typy interfejsów VR (2D, 3D, diegetyczne, systemowe)}

Projektowanie interfejsu do VR może odbywać się na różne sposoby, różniące się stopniem integracji z wirtualnym środowiskiem i tym jak prezentowane są istotne informacje. Aby odpowiednio przygotować się do projektowania, należy przeanalizować rozwiązania stosowane w istniejących produktach. Należy zrozumieć, co twórcy gier uznali za ważniejsze, pełną immersję czy raczej stawiali na tradycyjne menu i lewitujące w przestrzeni wirtualnej panele. W pierwszej kolejności warto jednak poznać kilka podstawowych rodzajów interfejsów i czym się charakteryzują. Pozwoli to na lepsze zrozumienie jak i kiedy są one wykorzystywane.W niniejszej analizie przyjęto podział interfejsów na cztery główne typy:

Interfejs jako część świata gry, Klasyczny panel menu (Non-diegetic UI), Interfejs przestrzenny oraz Meta.
Pierwszy rodzaj polega na wpleceniu interfejsu w elementy już istniejące w świecie wirtualnym, użytkownik żeby go zobaczyć musi fizycznie skierować głowę na dany element. Drugi z kolei wykorzystuje tradycyjny panel menu, ktory możemy spotkać na stronach internetowych czy w grach na komputer. Elementy interfejsu nie są częścią świata gry a sam panel jest nakładką na widok użytkownika. Kolejne podejście to interfejs przestrzenny, jest on połączeniem digetycznych i niedigetycznych.  Stanowi często część środowiska wirtualnego i jest w nim wyświtlany ale nie jest widoczny przez postacie. Ostatni typ to Meta, któy służy do reprezentacji statusu naszej postaci nie pojawiając się jednocześnie w świecie gry
%Interfejs diegetyczny – wpleciony w świat gry; użytkownik musi fizycznie spojrzeć na dany obiekt, aby zobaczyć informacje (np. zegarek na nadgarstku postaci).
%Interfejs niediegetyczny (klasyczny panel menu) – elementy UI nie są częścią świata gry, lecz nakładką na widok użytkownika (np. menu pauzy).
%Interfejs przestrzenny – zakotwiczony w przestrzeni wirtualnej, ale niekoniecznie istniejący w świecie gry z perspektywy postaci (np. menu przypięte do ściany).
%Interfejs statusowy (HUD) – reprezentuje stan postaci (np. poziom zdrowia), nie pojawiając się fizycznie w świecie gry.



\subsection{Przegląd interfejsów użytkownika w środowiskach VR(Najczęściej spotykane UI w VR}

Najczęsciej stosowane są interfejsy niedigetyczne






\subsection{Opis typowych sposobów interakcji w VR}
typy interakcji z VR
Siedzący (Seated VR)najczęściej stosowany w symulatorach lotniczych i wyścigowych, które wymagają precyzyjnego sterowania; stojący zwany rownież stacjonarnym
Stacjonarny 
Swobodny ruch (Room-scale VR)
Tryby renderowania i wyświetlania VR
Tryby śledzenia ruchu użytkownika
Tryby użytkowania w zależności od platformy
Tryby użytkowania VR według zastosowania
Czym jest Interfejs uzytkownika w VR
Czym są interakcje i jakie wyroznia sie metody interakcji w vr

\subsection{Klasyczne zasady UX}

\subsubsection{Hierarchia wizualna}
\subsubsection{Zasady Gestalt}
\subsubsection{Minimalizm i redukcja obciążenia poznawczego}
\subsubsection{Spójność i przewidywalność}
\subsubsection{Affordance i feedback}
\subsubsection{Heurystyki Nielsena}
\subsubsection{Róznice w projektowaniu interfejsu VR}
Chociaż większość podstawowych zasad UX designu jest uznawana jako uniwersalana to projektowanie interfejsu użytkownika w oparciu o nie wymaga od projektanta innego podejścia. Wynika to z odmienności systemu VR od innych technologii, takich jak aplikacje mobilne, aplikacje desktopowe czy strony internetowe, gdzie wszystkie interakcje z systemem odbywają się za pomocą myszki, klawiatury, ekranów dotykowych ewentualnie czytników ekranów. Tworzenie interfejsów użytkownika w środowisku wirtualnej rzeczywistości wiąże się z wieloma trudnościami, które nie występują w tradycyjnych aplikacjach. W projekcie konieczne będzie dostosowanie zasad tak, aby umożliwić użytkownikom komfortową, intuicyjną oraz płynną interakcję z produktem. Przy projektowaniu należy wziąć pod uwagę takie aspekty jak ograniczenia technologiczne i sensoryczne jak np. pole widzenia czy ograniczenia motoryczne użytkowników. 

Pierwszą fundamentalną różnicą między tradycyjnym wykorzystaniem zasad UX designu jest \textbf{Nawigacja}. W przypadku stron internetowych i aplikacji przemieszczanie się po produkcie odbywa się za pomocą klikania ikon, przewijania, ruchów myszką czy naciskania odpowiednich przycisków na klawiaturze. W wirtualnej rzeczywistości nawigowanie polega w głównej mierze na ruchach użytkownika szczególnie w przypadku bardziej zaawansowanych urządzeń takich jak  Pico 4 Ultra Enterprise, które nie wymagają dodatkowych urządzeń a korzystanie z nich odbywa się bezprzewodowo.(https://vr-expert.com/pl/samodzielne-vr/). Przemieszczanie po systemie odbywa się za pomocą śledzenia ruchow głową i ciała za pomocą kontrolerów ruchów. Często stosowane są również abstrakcyjne metody przemieszczania się, jak np. teleportacja do innego miejsca oddalonego w przestrzeni, co umożliwia eksplorowanie większych środowisk VR (Jennifer Whyte Dragana Nikolić - Virtual Reality and the Built Environment-Routledge (2018). Sam interfejs jest też inaczej umieszczony w przestrzeni. Tradycyjnie osadzone są w przestrzeni ekranu. W wirtualnej rzeczywistości interfejsy są umieszczone w przestrzeni  świata. Elementy mogą lewitować przed użytkownikiem lub częscią być częścią otoczenia. (Jonathan Linowes - Unity Virtual Reality Projects - Second Edition)
Następna równie ważna różnica to \textbf{Interakcja w przestrzeni 3D}. W tradycyjnych interfejsach  użytkownik wchodzi w interakcje z płaskimi, dwuwymiarowymi elementami na ekranie. W VR interakcje odbywają się w przestrzeni trójwymiarowej a użytkownik może dowolnie manipulować  obiektami tak jakby były on prawdziwe. Umożliwia chwytanie obracanie i przesuwanie element za pomocą rąk oraz interakcje które wymagają pełnego zaangażowania ciała użytkownika jak np. schylanie czy kucanie co wpływa pozytywnie na immersje i realizm doświadczenia.
Pozwala również na odejście od zgodności z rzeczywistością(nie wiem czy  to dobry synonim dla realizmu) i umożliwia działania niemożliwe w prawdziwym świecie jak przemieszczanie przedmiotów oddalonych w dużej odległości czy teleportacja. (Virtual, Augmented and Mixed Reality)
\textbf{Projektowanie interakcji} w wirtualnej rzeczywistości wymaga naśladowania w jaki sposob użytkownicy wchodzą w interakcje z przedmiotami i otoczeniem w prawdziwym świecie. Gesty i ruchy powinny być realistycznie odwzorowane a odpowiedz systemu natychmiastowa. Tradycyjne interakcje są bardziej pośrednie, gdy użytkownik chce wykonać jakieś działanie musi poruszyć myszką aby na ekranie kliknąć w pożądaną ikonę. W VR możliwe jest fizyczne "podniesienie" przedmiotu lub otwarcie drzwi.(Multimedia and Virtual Reality - Alistair Sutcliffe)  


\subsubsection{Adaptacja klasycznych zasad UX do środowiska VR}
\textbf{Czego Unikać}

Przeładowania informacjami
W VR mniej znaczy więcej, nadmierna liczba elementów w interfejsie może przytłoczyć użytkownika nadmiarem informacji i zaburzać immersję. Projektowanie do wirtualnego środowiska powinno skupiać się na minimalizmie i ograniczać do najważniejszych funkcji mniej istotne pomijając lub ukrywając. Redukcja zbędnych informacji i skupienie się na najistotniejszych funkcjonalnościach pomaga utrzymać koncentracje uzytkownika i ułatwia szybkie podejmowanie decyzji.

Nagłe zmiany perspektywy i ruchu kamery
Gwałtowne zmiany widoku mogą wywoływać dezorientacje, zawroty głowy a nawet mdłości. Poruszanie kamerą powinno odbywać się płynnie bez żadnych zakłóceń  naśladując naturalne ruchy głową użytkownika. Warto również uwzględnić stopniowe przejścia i animacje, aby zminimalizować negatywne skutki uboczne korzystania z VR. Stabilizacja i przewidywalne ruchy pomagają utrzymać komfort przez dłuższy czas. 

Brak standaryzacji interfejsów
Mimo dużej swobody w projektowaniu VR, brak spójnych i powtarzalnych zasad może frustrować użytkownikow. Należy stosować ujednolicone schematy interakcji tam gdzie to możliwe. Spojność ułatwia adaptacje do nowego środowiska, skraca to czas nauki obsługi sytemu oraz poprawia doświadczenia użytkownika. %

\subsection{Metody UX stosowane w projekcie}


\section{Tło technologiczne VR} 

Kluczowym etapem projektowania tej pracy był wybór odpowiednich narzędzi umożliwiających prototypownie, implementację oraz testowanie interfejsu w środowisku VR. Wybór używanej technologii jest kluczowy, aby zapewnić wysoką jakość doświadczeń użytkownika oraz zachować efektywność samego procesu projektowania.


\subsection{Silniki gier i środowiska VR (Unity, Unreal, Godot)}

\textit{Unity} to jeden z najpopularniejszych wieloplatformowych silników do tworzenia gier wideo oraz interaktywnych aplikacji. Umożliwia zaawansowaną obsługę grafiki i fizyki, a w samym edytorze udostępniono rozbudowany zestaw narzędzi do projektowania scen, animacji i efektów graficznych \ref{unity_engine_example}. Proces programowania odbywa się przy użyciu języka \textit{C\#}.

W Unity dodano również wsparcie dla wirtualnej oraz rozszerzonej rzeczywistości, co przekłada się na szerokie zastosowanie tej technologii w szkoleniach, symulacjach czy działaniach marketingowych. Oprócz możliwości tworzenia zaawansowanych pod względem wizualnym scen, istotna jest też opcja łatwej integracji gotowych wtyczek obsługujących urządzenia AR i VR od różnych producentów. Rozbudowany system oświetlenia oraz post-processingu zapewnia szeroki wachlarz opcji i umożliwia dostosowanie grafiki pod dedykowane urządzenia, dzięki czemu można uzyskać zadowalającą jakość i odpowiednią optymalizację na urządzenia mobilne oraz wysoce realistyczną grafikę na komputerach osobistych.

\begin{figure}[!htb]
    \centering
    \includegraphics[width=0.8\textwidth]{images/unity.png}
    \caption{Przykładowy wygląd edytora Unity}
    \label{unity_engine_example}
\end{figure}

Regularne aktualizacje silnika rozszerzają jego funkcjonalność i dodają nowe rozwiązania, takie jak obsługa ray tracingu, DLSS oraz nowe narzędzia. Stały rozwój sprawia, że Unity zachowuje elastyczność w obliczu zmieniających się potrzeb rynku, a jego wszechstronność umożliwia realizację nawet najbardziej rozbudowanych koncepcji interaktywnych.

Społeczność skupiona wokół tego silnika udostępnia liczne materiały edukacyjne w postaci kursów i filmów, co znacząco obniża poziom wejścia, przyspiesza naukę oraz ułatwia rozwiązywanie ewentualnych problemów technicznych. Dokumentacja silnika jest regularnie rozwijana, a oficjalny sklep \textit{Asset Store} zapewnia dostęp do bardzo dużej ilości bezpłatnych oraz płatnych pakietów, obejmujących zasoby takie jak modele, tekstury i dźwięki oraz dodatkowe użyteczne narzędzia wraz z całymi gotowymi systemami ułatwiającymi stworzenie projektu i pozwalającymi ograniczyć koszty produkcji.


\subsection{Biblioteki i frameworki VR (XR Toolkit, OpenXR, SteamVR)}

\subsection{Narzędzia do projektowania interfejsów (Figma, Blender)}

Figma to podstawowe narzędzie służące do projektowania interfejsów użytkownika, jednocześnie umożliwia tworzenie interaktywnych klikalnych prototypów aplikacji. Jest to aktualnie jedno z najbardziej popularnych i bezpłatnych narzędzi w branży UX/UI. W projekcie Figma zostanie wykorzystana do stworzenia wstępnych projektów i prototypów interfejsów, a następnie do przetestowania rożnych układów elementów interfejsu. 

\subsection{Narzędzia do analizy UX/UI (Google Forms, UEQ, SUS)}

Gogle VR to urządzenia, które całkowicie zasłaniają pole widzenia i wyświetlają przez użytkownikiem dwa identyczne, ale nieco przesunięte od siebie obrazy, które nałożone na siebie przez mózg dają wrażenie głębi. Wewnątrz gogli umieszczone są specjalne czujniki takie jak żyroskop, kamery i inne śledzące położenie i ruch głowy w przestrzeni. Dzięki temu użytkownik obracając głową faktycznie rozgląda się w przestrzeni wirtualnej obracając kamerą gracza.

Oprócz gogli istotne są również specjalne kontrolery zakładane na dłonie. Dzięki nim możliwa jest interakcja z wirtualnym otoczeniem, a specjalne czujniki i przyciski wykrywają pozycje palców i pozwalają określić, czy gracz w tej chwili próbuje złapać przedmiot. Całość wraz z systemem śledzenia pozycji dłoni pozwala dowolnie sięgać w różnych kierunkach, co pozwala mieć wpływ na obiekty w aplikacji. Gracz może na przykład chwycić przedmiot i rzucić nim, co pozwoli wywołać kolejne symulacje w fizyce.

Oprócz tego istnieje wiele różnych dodatków rozszerzających możliwości wirtualnej rzeczywistości, jak specjalne kombinezony monitorujących ruch oraz umożliwiających odczuwanie na skórze poprzez elektrostymulację nerwów i mięśni lub kapsuły, w których użytkownik ma możliwość skakania oraz poruszania się w dowolny sposób bez ryzyka, że uszkodzi coś w pokoju.

2.9. Sprzęt i konfiguracja testowa

