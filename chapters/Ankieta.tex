\chapter{Ankieta}
\section{Na ten momement tutaj bo jeszcze nie wiem gdzie ją umieścić}
Opracowanie Ankiety skierowanej do użytkowników mających doświadczenie z Virtualną Rzeczywistością było spowodowane potrzebą zidentyfikowania problemów pojawiających się w już istniejących Grach VR jeszcze przed rozpoczęciem etapu projektowania. Ankieta ma na celu zebranie wiarygodnych danych na temat pozytywnych i negatywnych doświadczeń użytkwoników z grami VR oraz ma pozwolić na określenie które elementy UI były dla badanych nieintuicyjne lub męczące. Badanie będzie miało charakter diagnostyczny i będzie pierwszym etapem analizy potrzeb użytkowników w celu określenia tzw. Punktów bólu (ang. pain points czyli miejsc, w których doświadczenie immersji lub ergonomia interfejsu ulegają zaburzeniu. Zebrane dane posłużą w dalszej części jako podstawa do opracowania wytycznych projektowych i posłuża w podejmowaniu decyzji projektowych.

Wybór grupy docelowej był bardzo istotny, osoby doświadczone mają inne kompetencje poznawcze i techniczne niż osoby nie mające styczności z VR. Dzieki obcowaniu z produktami są bardziej świadomi możliwości jakie daje wirtualna rzeczywistość i ograniczeń a dzięki styczności z różnymi typami gier i aplikacji są zorientowani na elementy które zaburzają ich immersje i zmniejszają komfort z użytkowania. Osoby początkujące skupiają się bardziej na ogólnym wrażeniu immersji, podczas pierwszego użytkowania mogą być przytłoczeni ilością bodzców więc mogą nie zwrócić uwagę na takie szczegóły jak osoby doświadczone, te z kolei zauważają takie detale jak precyzjne wykonywanie gestów,płynność interakcji czy stabilność systemu. Niewłaściwa architektura informacji  czy błedne rozmieszczenie elementów może spowodować u takiego gracza trudność w znalezieniu pożądanego elementu a to z kolei irytacje. Poznanie ich perspektywy pozwoli już na tak wczesnym etapie wykluczyć napotkane przez nich punkty bólu oraz znacząco wpłynie na ergonomie i użytecznoś interfejsu. 

Założenia konstrukcyjne
Z literatury wynika, że poprawnie zaprojektowane narzędzie badawcze będzie miało ogromny wpływ na jakość pozyskanych danych.

Czas trwania	ok. 12 minut
Forma	ankieta online (Google Forms)
Typ pytań	skala Likerta 1–7 + 3 pytania otwarte
Liczba pytań	20–22 zamknięte + 3 otwarte
Cel	ocenić jakość, immersję i ergonomię interfejsów VR z perspektywy użytkownika doświadczonego

%Pytania jeszcze raz muszę dopracować nadal są mało precyzyjne. Myślę żeby podzielić je na mniejszą ilość sekcji.

Struktura ankiety 


Sekcja 1. Wprowadzenie 
Jak często korzystasz z systemów VR?
Jakie aplikacje i gry VR używasz najczęściej (rozrywka, symulacje, edukacja, projektowanie)?
Sekcja 2. Immersja i poczucie obecności 
Na ile czułeś się ‘obecny’ w środowisku VR?
Czy elementy UI wpływały na Twoje poczucie immersji pozytywnie czy negatywnie?
Czy interfejs był zintegrowany z otoczeniem (np. diegetyczny)?
Sekcja 3. Interakcja i kontrola 
Czy reakcje systemu były zgodne z Twoimi oczekiwaniami ruchowymi?
Jak oceniasz precyzję manipulacji w interfejsie?
Czy układ elementów był logiczny i przewidywalny?
Sekcja 4. Komfort i ergonomia (2–3 pytania)
Czy elementy interfejsu wymagały nadmiernego wysiłku fizycznego lub ruchu?
Czy interfejs utrudniał Ci koncentrację lub powodował zmęczenie oczu?
Sekcja 5. Wydajność i informacja zwrotna
Czy informacje zwrotne (wizualne, dźwiękowe, haptyczne) były wystarczające?
Czy opóźnienia w reakcji interfejsu wpływały na Twoje doświadczenie?
Sekcja 6. Pytania otwarte 
Co najbardziej przeszkadzało Ci w dotychczasowych interfejsach VR?
Jakie rozwiązania uznajesz za najlepsze w VR UI, które testowałeś?
Jakie elementy Twoim zdaniem najbardziej wpływają na immersję i komfort UI w VR?

5. Dobór i pilotaż

– rekrutacja min. 10–20 uczestników
– test wstępny (pilot) w środowisku podobnym do docelowego,
– analiza zrozumiałości pytań i długości, 
% Mam problem ze słownictwem bo w teori takie słowa jak haptyczny czy digetyczny to dosyć podstawowe słowo w zakresie vr ale pytałam kilka osób poglądowo(8) i nie znają znaczenia niektórych tego typu słów mimo bardzo dużego doświadczenia w grach (równnież VR).
– ewentualna modyfikacja

%Poprawki - słownictwo, stuktura ogólna i podział na sekcje, dopisanie źródeł informacji. przeredagowanie tekstu