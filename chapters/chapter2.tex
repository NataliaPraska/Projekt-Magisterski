\chapter{Podstawowe Zasady UX/UI designu}

\section{Wprowadzenie do UX Designu}

 UX design, czyli projektowanie doświadczeń użytkownika, to proces tworzenia zapewniający pozytywne doświadczenia użytkownikom poprzez optymalizacje interakcji z daną stroną, systemem czy produktem. Początkowo zagadnienie to było kojarzone głównie z projektowaniem stron internetowych i aplikacji, jednak współczesne podejście znajduje zastosowanie w każdej dziedzinie. Obecnie UX design obejmuje projektowanie przeróżnych produktów, od stron internetowych po urządzenia codziennego użytku jak urządzenia AGD, samochody czy nawet meble. Aktualnie UX odnosi się nie tylko do wirtualnych interfejsów ale także do interakcji ze zwykłymi przedmiotami z którymi styczność mamy wszyscy.

 \section{Znaczenie UX w procesie projektowania}

 W sytuacji, gdy użytkownik napotyka trudności w interakcji z przedmiotem, takim jak drzwi, i nie wie, jak wykonać określoną czynność (np. otworzyć drzwi), problemem nie jest sam użytkownik, lecz projekt produktu(Norman, 2013).


\section{Podstawowe Zasady UX/UI i ich zastosowanie w VR}

W kontekście VR projektowanie interfejsów staje się szczególnie problematyczne ze względu na konieczność osiągnięcia równowagi między imersją w wirtualnej rzeczywistości a zapewnieniem intuicyjnej interakcji oraz komfortu użytkowania dlatego istotne jest zastosowanie sprawdzonych zasad UX/UI, które umożliwią stworzenie rozwiązania dla specyfiki tej technologii.
Z ksiązek "The Design of Everyday Things" (Norman, 2013), "Don't Make Me Think" (Krug, 2014), "The Art and Science of UX Design" (Pattiwal, 2024) oraz "The Rules of UX Design" (Pattiwal, 2024) autorzy wyróżniają szereg fundamentalnych zasad UX designu, które pomagają w tworzeniu bardziej efektywnych, intuicyjnych i przyjemnych doświadczeń użytkowników.
Wśród kluczowych zasad projektowania UX/UI wyróżnia się:

\subsection{Ergonomia} – Produkt powinien być projektowany tak, by był wygodny i łatwy w użyciu. Dobre projektowanie UX uwzględnia nie tylko interakcje użytkownika z urządzeniem, ale również sposób, w jaki przetwarza on informacje wyświetlane na ekranie. Celem jest umożliwienie użytkownikowi szybkiego osiągania zamierzonych celów, bez nadmiernego wysiłku (Don Norman, 2013).

\subsection{Intuicyjność} – Interfejsy muszą być zaprojektowane tak aby użytkownicy nie musieli się zastanawiać, a jaki sposób wykonać określoną czynność. Powinny być łatwe do zrozumienia a użytkownicy powinni od razu wiedzieć, jak poruszać się po systemie (Krug, 2014).

\subsection{Spójność} - wszystkie elementy interfejsu muszą działać w podobny sposób w rożnych częściach systemu. Elementy interfejsu muszą być spójnie rozmieszczone co ułatwi użytkownikom poruszanie się po aplikacji
Intuicyjność i immersja w VR

\subsection{Minimalizm} 

\subsection{Zrozumienie kontekstu użytkownika}

\subsection{Feedback}

\subsection{Widoczność i dostępność}

\subsection{Prostota w interakcji}