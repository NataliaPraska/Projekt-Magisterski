\chapter{Wprowadzenie do tematyki pracy, Analiza rozwiązań dostępnych na rynku, wykorzystane narzędzia}

\section {Przegląd aktualnych technologii VR}
\section {Analiza istniejących aplikacji VR w kontekście interakcji użytkownika}
\section {Identyfikacja najlepszych praktyk i innowacji w projektowaniu interfejsow VR}

Pierwszym etapem projektowania tej pracy był wybór odpowiednich narzędzi umożliwiających prototypownie, implementację oraz testowanie interfejsu w środowisku VR. Wybór używanej technologii jest kluczowy, aby zapewnić wysoką jakość doświadczeń użytkownika oraz zachować efektywność samego procesu projektowania.

\subsection{Figma}
Figma to podstawowe narzędzie służące do projektowania interfejsów użytkownika, jednocześnie umożliwia tworzenie interaktywnych klikalnych prototypów aplikacji. Jest to aktualnie jedno z najbardziej popularnych i bezpłatnych narzędzi w branży UX/UI. W projekcie Figma zostanie wykorzystana do stworzenia wstępnych projektów i prototypów interfejsów, a następnie do przetestowania rożnych układów elementów interfejsu. 
%(tu będzie dokumentacja figmy) 
\subsection{Unity}
\subsection{Sprzęt VR}
\subsection{Google Forms i inne narzędzia do analizy UX/UI}








%Tworzenie interfejsów użytkownika w środowisku wirtualnej rzeczywistości wiąże się z wieloma trudnościami, które nie występują w tradycyjnych aplikacjach. Ze względu na trójwymiarowy wymiar wirtualnej rzeczywistości konieczne będzie zastosowanie nowych zasad ergonomii, ktore umożliwią użytkownikom komfortową, intuicyjną oraz płynną interakcję z produktem. Przy projektowaniu należy wziąć pod uwagę takie aspekty jak ograniczenia technologiczne i sensoryczne jak np. pole widzenia czy ograniczenia motoryczne użytkowników. 

\section{Ogólne zasady projektowania interfejsów użytkownika}