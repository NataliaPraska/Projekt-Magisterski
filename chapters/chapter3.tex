\chapter{Analiza istniejących już na rynku rozwiązań}

% Analiza istniejących już na rynku rozwiązań

Projektowanie interfejsu do VR może odbywać się na kilka różnych sposobów różniących się od siebie sposobem integracji z wirtualnym środowiskiem i tym jak prezentowane są istotne informacje. Aby odpowiednio przygotować się do projektowania, należy zapoznać się z najczęściej stosowanymi rozwiązaniami w istniejących już na rynku produktach. Należy zrozumieć, co twórcy gier i symulatorów uznali za ważniejsze, pełną immersję czy raczej stawiali na tradycyjne menu i lewitujące w przestrzeni wirtualnej panele. W pierwszej kolejności warto jednak poznać kilka podstawowych rodzajów interfejsów i czym się charakteryzują. Pozwoli to na lepsze zrozumienie jak i kiedy są one wykorzystywane. Na potrzeby analizy zastosowano podział interfejsów na cztery najczęściej stosowane rodzaje. Interfejs jako część świata gry, Klasyczny panel menu (Non-diegetic UI), Interfejs panelowy w środowisku VR (Spatial UI) oraz HUD (Head-Up Display.
Pierwszy rodzaj polega na wpleceniu interfejsu w elementy już istniejące w świecie wirtualnym, użytkownik żeby go zobaczyć musi fizycznie skierować głowę na dany element. Drugi z kolei wykorzystuje tradycyjny panel menu, ktory możemy spotkać na stronach internetowych czy w grach na komputer. Elementy interfejsu nie są częścią otoczenia a sam panel jest nakładką na widok użytkownika. Kolejne podejście to interfejs panelowy ale umieszczony w środowisku wirtualnym. Typ ten charakteryzuje się unoszącym się w powietrzu menu, które może być sterowane za pomocą kontrolera lub ruchami rąk. Ostatni typ to HUD, rodzaj stosowany często w tradycyjnych grach. Najważniejsze informacje są "przyklejone" do widoku gracza, pozostając cały czas w jego polu widzenia niezależnie od miejsca w którym znajduje się użytkownik w wirtualnym otoczeniu.

\subsection{Half-Life: Alyx}

(Jonathan Linowes w Unity Virtual Reality Projects)
\subsection{Beat Saber}
\subsection{Phasmophobia VR}
\subsection{No Man's Sky}
https://ekspert.ceneo.pl/najlepsze-gry-vr
\subsection{Symulator VR „Zaawansowane procedury medyczne”
}
"Rozwija praktyczne umiejętności studentów w zakresie segregacji medycznej, udzielania kwalifikowanej pierwszej pomocy, podstawowej pierwszej pomocy oraz ratownictwa medycznego. Posiada edytor umożliwiający wybór odpowiedniego środowiska, konfigurację pacjentów, dostępnego sprzętu i wartości referencyjnych. Dostępny jest w trybach: egzaminacyjnym lub ćwiczeniowym; jedno bądź wieloosobowym oraz w wariancie PC i VR."


https://uxdesign.cc/vr-diegetic-interfaces-dont-break-the-experience-554f210b6e46




