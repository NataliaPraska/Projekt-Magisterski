\chapter{Projektowanie interfejsu VR}


\section{Analiza rozwiązań istniejących już na rynku}
Projektowanie interfejsu do VR może odbywać się na kilka różnych sposobow w zależności, rożniących się od siebie sposobem integracji z wirtualnym środowiskiem i tym jak prezentowane są istotne informacje. Aby odpowiednio przygotować się do projektowania, należy zapoznać się z istniejącymi już na rynku produktami. Należy zapoznać się co twórcy gier i symulatorow uznali za najważniejsze, pełną immersje czy raczej stawiali na tradycyjne menu i lewitujące w przestrzeni wirtualnej panele. Na potrzeby analizy tych rozwiązań zastosowano podział na 4 najczęściej stosowane rodzaje interfejsow

\subsection{Interfejs jako cześć świata gry}


No Man’s Sky VR
The Walking Dead: Saints & Sinners
Half-Life: Alyx

(Jonathan Linowes w Unity Virtual Reality Projects)



\subsection{Klasyczny panel menu (Non-diegetic UI)}
Phasmophobia VR
Beat Saber

\subsection{Interfejs panelowy w środowisku VR (Spatial UI)}
No Man’s Sky VR
Tilt Brush 

\subsection{HUD (Head-Up Display}
War Thunder VR
Elite Dangerous VR

https://uxdesign.cc/vr-diegetic-interfaces-dont-break-the-experience-554f210b6e46


\section{Tworzenie prototypów interfejsów użytkownika dla VR}
Aby uniknąć wielokrotnej implementacji niezliczonej ilości ilości interfejsów 
\section{Testowanie różnych układów elementów i metod interakcji}
%(kontrolery, gesty, głos).
\section{Projektowanie interaktywnych obiektów}
\subsection{obiekt fizyczny dynamiczny}
\subsection{obiekt fizyczny statyczny}
\subsection{obiekt wirtualny}
\section{Tworzenie elementów wizualnych}
%Tworzenie atrakcyjnych wizualnie i funkcjonalnych przycisków.


