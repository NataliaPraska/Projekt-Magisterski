\chapter{Badania wstępne}


\section{Cel i znaczenie badań UX}
Celem przeprowadzenia ankiety było zebranie opinii użytkowników posiadających doświadczenie w korzystaniu z technologii wirtualnej rzeczywistości oraz ocena ich subiektywnych doświadczeń związanych z grami VR. Dobór respondentów z co najmniej podstawowym doświadczeniem w VR wynikał z założenia, że nawet jednorazowy kontakt z technologią pozwala użytkownikowi zidentyfikować elementy powodujące dyskomfort, frustrację lub prowadzące do rezygnacji z dalszego korzystania z aplikacji VR. Badanie zostało zaplanowane jako etap wstępny, realizowany przed rozpoczęciem projektowania i implementacji własnego rozwiązania, lecz po przeprowadzeniu analizy istniejących gier dostępnych na rynku. Jego zadaniem było potwierdzenie problemów zaobserwowanych podczas tej analizy oraz sprawdzenie, w jakim stopniu pokrywają się one z rzeczywistymi doświadczeniami użytkowników. W pracy pojęcie punktów bólu odnosi się do problemów, frustracji oraz niedogodności, na jakie użytkownicy napotykają podczas interakcji z grami VR, w szczególności w obszarze interfejsu użytkownika oraz mechanizmów interakcji. Ankieta miała na celu zarówno potwierdzenie wcześniej zidentyfikowanych punktów bólu, jak i umożliwienie ujawnienia dodatkowych problemów, które mogły nie zostać dostrzeżone na etapie analizy wybranych tytułów.

Zebrane dane stanowią podstawę do dalszych decyzji projektowych i posłużą do sformułowania założeń dotyczących projektowania interfejsu oraz interakcji w prototypie gry VR, ze szczególnym uwzględnieniem elementów wpływających na komfort użytkowania i poziom immersji.

\section{Pytania badawcze}

Na potrzeby badań wstępnych sformułowano pytania badawcze odnoszące się do obszarów, które w praktyce najczęściej wpływają na komfort korzystania z gier VR oraz na utrzymanie immersji. Pytania te mają charakter eksploracyjny i mogą ulec doprecyzowaniu po analizie wyników ankiety.
Sformułowane pytania badawcze:

\begin{enumerate}
    \item Jakie elementy interfejsu użytkownika w grach VR są postrzegane jako najbardziej zakłócające poczucie immersji i obecności?
    \item Jak intuicyjność, precyzja i sposób interakcji oraz układ interfejsu wpływają na doświadczenie użytkownika w VR?
     \item Jakie elementy interfejsu VR powodują dyskomfort fizyczny lub percepcyjny u użytkowników?
     \item Jak brak lub opóźnienie informacji zwrotnej w interfejsie VR wpływa na immersję i komfort użytkownika?

\end{enumerate}

Pytania ankietowe zostały opracowane w oparciu o sformułowane pytania badawcze. Szczegółowy opis struktury ankiety, podziału na sekcje tematyczne oraz rodzaju zastosowanych pytań przedstawiono w podrozdziale 4.4
Spośród sformułowanych pytań badawczych kluczowe znaczenie miało pytanie dotyczące elementów interfejsu użytkownika zakłócających poczucie immersji i obecności. Stanowi ono bezpośrednie uzasadnienie podjęcia tematu pracy i punkt wyjścia do dalszych analiz projektowych. Pozostałe pytania pełnią funkcję uzupełniającą i pozwalają lepiej zrozumieć charakter zidentyfikowanych problemów, tak aby projektowany interfejs mógł zostać możliwie najlepiej dopasowany do specyfiki analizowanego typu gry VR.

Należy podkreślić, że pytania badawcze zostały sformułowane jako pytania eksploracyjne, których celem jest identyfikacja powtarzających się problemów oraz zebranie informacji wspierających proces podejmowania decyzji projektowych na etapie tworzenia prototypu


\section{Metodyka badań}
W ramach badań wstępnych zastosowano podejście łączące analize heurystyczną i badanie ankietowe.
Analiza wybranych gier VR, przeprowadzona z wykorzystaniem heurystyk użyteczności dla rzeczywistości wirtualnej opracowanych przez Nielsen Norman Group (2021) była oparta na obserwacji i pozwoliła mi na wstepną identyfikacje potencjalnych problemów w projektowaniu interfejsów. Nie umożliwiała jednak jednoznacznej weryfikacji, które z tych zagadnień rzeczywiście wpływają na subiektywne doświadczenie użytkownika.

Z tego względu uzupelniono ją badaniem ankietwoym skierowanym do osób juz posiadających choć niewielkie doświadczenie w korzystaniu z technologi VR. Zastosowanie ankiety pozwoliło na zebranie opinii użytkowników oraz potwierdzenie istotności wcześniej zidentyfikowanych problemow z perspektywy ich rzeczywistych doświadczeń. Takie połączenie metod umożliwiło zestwienie obserwacji projektowych z opiniami użytkowników i stanowiło podstawę do dalszych decyzji projektowych.

\section{Konstrukcja ankiety, grupa badawcza i sposób dystrybucji}
% Do rozwiniecia

Ankieta została zaprojektowana jako autorskie narzędzie badawcze, którego konstrukcja wynikała bezpośrednio z wcześniej sformułowanych pytań badawczych. W celu ułatwienia respondentom koncentracji pytania zostały pogrupowane w sekcje tematyczne. Miało to zwiększyć czytelność ankiety oraz ułatwiałć zrozumienie kontekstu poszczególnych zagadnień. Pierwsza część ankiety miała charakter wprowadzający i dotyczyła ogólnego doświadczenia respondentów z technologią VR, w tym częstotliwości korzystania z wirtualnej rzeczywistości oraz kontekstu jej użycia. Informacje te pozwalały określić poziom obycia respondentów z technologią i stanowiły punkt odniesienia dla interpretacji dalszych odpowiedzi.
Kolejne pytania koncentrowały się na immersji i poczuciu obecności w środowisku VR oraz na sposobach interakcji i kontroli. Respondenci oceniali wpływ elementów interfejsu na ciągłość doświadczenia, intuicyjność wykonywanych gestów, precyzję rozpoznawania ruchów oraz czytelność układu menu i elementów sterowania.
Następna sekcja tematyczna pytań dotyczyła komfortu i ergonomii użytkowania oraz informacji zwrotnej i responsywności systemu. Uwzględniono odczuwany wysiłek fizyczny, zmęczenie, dezorientację, a także czytelność sygnałów wizualnych i dźwiękowych oraz wpływ ewentualnych opóźnień reakcji interfejsu na komfort użytkowania.
W ankiecie zadecydowano o zastosowaniu pytania zamknięte w formie stwierdzeń ocenianych w pięciostopniowej skali od „zdecydowanie się nie zgadzam” do „zdecydowanie się zgadzam”, co pozwalało określić, czy dany aspekt interfejsu był postrzegany jako problematyczny. Uzupełnieniem do nich były pytania otwarte, które umożliwiały opisanie problemów oraz pozytywnych praktyk interfejsowych, w tym sprawdzenie, czy użytkownicy wskazują te same elementy interfejsu, które zostały wcześniej wyróznione podczas analizy istniejących gier, bez sugerowania odpowiedzi z góry. Czas wypełniania ankiety ograniczono do około piętnastu minut. Miało to na celu utrzymanie zaangażowania respondentów oraz ograniczenie ryzyka udzielania odpowiedzi w sposób nieuważny.
Grupę badawczą stanowili użytkownicy posiadający jakiekolwiek doświadczenie w korzystaniu z technologii wirtualnej rzeczywistości, od jednorazowego kontaktu po regularne użytkowanie. Ujęcie respondentów o zróżnicowanym poziomie doświadczenia pozwalało uwzględnić zarówno perspektywę pierwszego kontaktu z VR, jak i bardziej świadome, porównawcze spojrzenie na interfejs i mechanizmy interakcji.
Osoby początkujące dostarczały informacji dotyczących progu wejścia w doświadczenie VR oraz elementów interfejsu postrzeganych jako niezrozumiałe lub zniechęcające. Z kolei użytkownicy bardziej doświadczeni wskazywali problemy związane z precyzją gestów, płynnością interakcji, stabilnością systemu oraz architekturą informacji. Zestawienie tych perspektyw umożliwiało pełniejsze zidentyfikowanie punktów bólu istotnych z punktu widzenia projektowania interfejsu.
Ankieta była dystrybuowana w sposób celowy poprzez bezpośrednie udostępnienie jej osobom potencjalnie spełniającym kryteria badania. Link do ankiety przekazano za pośrednictwem wiadomości prywatnych na platformie Discord oraz udostępniono wśród znajomych, rodziny i w środowisku uczelnianym, co umożliwiło sprawne zebranie danych od respondentów o zróżnicowanym poziomie doświadczenia z technologią VR.

\section{Analiza wyników ankiety}

\section{Wnioski UX do etapu prototypowania}
