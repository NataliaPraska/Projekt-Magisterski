\chapter{Ankieta}
\section{Na ten momement tutaj bo jeszcze nie wiem gdzie ją umieścić}
Opracowanie Ankiety skierowanej do użytkowników mających doświadczenie z Virtualną Rzeczywistością było spowodowane potrzebą zidentyfikowania problemów pojawiających się w już istniejących Grach VR jeszcze przed rozpoczęciem etapu projektowania. Ankieta ma na celu zebranie wiarygodnych danych na temat pozytywnych i negatywnych doświadczeń użytkwoników z grami VR oraz ma pozwolić na określenie które elementy UI były dla badanych nieintuicyjne lub męczące. Badanie będzie miało charakter diagnostyczny i będzie pierwszym etapem analizy potrzeb użytkowników w celu określenia tzw. Punktów bólu (ang. pain points czyli miejsc, w których doświadczenie immersji lub ergonomia interfejsu ulegają zaburzeniu. Zebrane dane posłużą w dalszej części jako podstawa do opracowania wytycznych projektowych i posłuża w podejmowaniu decyzji projektowych.

Wybór grupy docelowej był bardzo istotny, osoby doświadczone mają inne kompetencje poznawcze i techniczne niż osoby nie mające styczności z VR. Dzieki obcowaniu z produktami są bardziej świadomi możliwości jakie daje wirtualna rzeczywistość i ograniczeń a dzięki styczności z różnymi typami gier i aplikacji są zorientowani na elementy które zaburzają ich immersje i zmniejszają komfort z użytkowania. Osoby początkujące skupiają się bardziej na ogólnym wrażeniu immersji, podczas pierwszego użytkowania mogą być przytłoczeni ilością bodzców więc mogą nie zwrócić uwagę na takie szczegóły jak osoby doświadczone, te z kolei zauważają takie detale jak precyzjne wykonywanie gestów,płynność interakcji czy stabilność systemu. Niewłaściwa architektura informacji  czy błedne rozmieszczenie elementów może spowodować u takiego gracza trudność w znalezieniu pożądanego elementu a to z kolei irytacje. Poznanie ich perspektywy pozwoli już na tak wczesnym etapie wykluczyć napotkane przez nich punkty bólu oraz znacząco wpłynie na ergonomie i użytecznoś interfejsu. 

Założenia konstrukcyjne
Z literatury wynika, że poprawnie zaprojektowane narzędzie badawcze będzie miało ogromny wpływ na jakość pozyskanych danych.

Czas trwania	ok. 12 minut
Forma	ankieta online (Google Forms)
Typ pytań	skala Likerta 1–7 + 3 pytania otwarte
Liczba pytań	20–22 zamknięte + 3 otwarte
Cel	ocenić jakość, immersję i ergonomię interfejsów VR z perspektywy użytkownika doświadczonego

%Pytania jeszcze raz muszę dopracować nadal są mało precyzyjne. Myślę żeby podzielić je na mniejszą ilość sekcji.



5. Dobór i pilotaż

– rekrutacja min. 10–20 uczestników
– test wstępny (pilot) w środowisku podobnym do docelowego,
– analiza zrozumiałości pytań i długości, 
% Mam problem ze słownictwem bo w teori takie słowa jak haptyczny czy digetyczny to dosyć podstawowe słowo w zakresie vr ale pytałam kilka osób poglądowo(8) i nie znają znaczenia niektórych tego typu słów mimo bardzo dużego doświadczenia w grach (równnież VR).
– ewentualna modyfikacja

%Poprawki - słownictwo, stuktura ogólna i podział na sekcje, dopisanie źródeł informacji. przeredagowanie tekstu


Stare notatki do ankiety 

\chapter{Badania wstępne}



\section{Cel i znaczenie badań UX}
Celem przeprowadzenia ankiety było zebranie opinii użytkowników posiadających doświadczenie w korzystaniu z technologii wirtualnej rzeczywistości oraz ocena ich subiektywnych doświadczeń związanych z grami VR. Dobór respondentów z co najmniej podstawowym doświadczeniem w VR wynikał z założenia, że nawet jednorazowy kontakt z technologią pozwala użytkownikowi zidentyfikować elementy powodujące dyskomfort, frustrację lub prowadzące do rezygnacji z dalszego korzystania z aplikacji VR.

Badanie zostało zaplanowane jako etap wstępny, realizowany przed rozpoczęciem projektowania i implementacji własnego rozwiązania, lecz po przeprowadzeniu analizy istniejących gier dostępnych na rynku. Jego zadaniem było potwierdzenie problemów zaobserwowanych podczas tej analizy oraz sprawdzenie, w jakim stopniu pokrywają się one z rzeczywistymi doświadczeniami użytkowników.

W pracy zastosowano pojęcie punktów bólu który odnosi się do problemów, frustracji oraz niedogodności, na jakie użytkownicy napotykają podczas interakcji z grami VR, w szczególności w obszarze interfejsu użytkownika oraz mechanizmów interakcji. Ankieta miała na celu zarówno potwierdzenie wcześniej zidentyfikowanych punktów bólu, jak i umożliwienie ujawnienia dodatkowych problemów, które mogły nie zostać dostrzeżone na etapie analizy wybranych tytułów.

Zebrane dane stanowią podstawę do dalszych decyzji projektowych i posłużą do sformułowania założeń dotyczących projektowania interfejsu oraz interakcji w prototypie gry VR, ze szczególnym uwzględnieniem elementów wpływających na komfort użytkowania i poziom immersji.

\section{Pytania badawcze}

\section{Metodyka badań}

\section{Konstrukcja ankiety}

Z literatury wynika, że poprawnie zaprojektowane narzędzie badawcze będzie miało ogromny wpływ na jakość pozyskanych danych.

Czas trwania	ok. 12 minut
Forma	ankieta online (Google Forms)
Typ pytań	skala Likerta 1–7 + 3 pytania otwarte
Liczba pytań	20–22 zamknięte + 3 otwarte
Cel	ocenić jakość, immersję i ergonomię interfejsów VR z perspektywy użytkownika doświadczonego

\section{Grupa badawcza i sposób dystrybucji}

Wybór grupy docelowej był bardzo istotny, osoby doświadczone mają inne kompetencje poznawcze i techniczne niż osoby nie mające styczności z VR. Dzieki obcowaniu z produktami są bardziej świadomi możliwości jakie daje wirtualna rzeczywistość i ograniczeń a dzięki styczności z różnymi typami gier i aplikacji są zorientowani na elementy które zaburzają ich immersje i zmniejszają komfort z użytkowania. Osoby początkujące skupiają się bardziej na ogólnym wrażeniu immersji, podczas pierwszego użytkowania mogą być przytłoczeni ilością bodzców więc mogą nie zwrócić uwagę na takie szczegóły jak osoby doświadczone, te z kolei zauważają takie detale jak precyzjne wykonywanie gestów,płynność interakcji czy stabilność systemu. Niewłaściwa architektura informacji  czy błedne rozmieszczenie elementów może spowodować u takiego gracza trudność w znalezieniu pożądanego elementu a to z kolei irytacje. Poznanie ich perspektywy pozwoli już na tak wczesnym etapie wykluczyć napotkane przez nich punkty bólu oraz znacząco wpłynie na ergonomie i użytecznoś interfejsu.

\section{Analiza wyników ankiety}

\section{Wnioski UX do etapu prototypowania}
