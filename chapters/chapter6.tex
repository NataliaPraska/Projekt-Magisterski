\section{Badania końcowe i podsumowanie pracy}

\subsection{Cel badań A/B}
 Celem przeprowadzenia badania A/B było empiryczne porównanie wpływu dwóch różnych podejść do projektowania interfejsu użytkownika oraz wybranych mechanizmów interakcji w wirtualnym środowisku. Analiza obejmowała typ interfejsu, sposób przemieszczania się oraz intensywność informacji zwrotnej, czyli liczbę kanałów komunikujących reakcje systemu na wykonane zadania. Miały również ocenić poziom immersji i poczucia obecności użytkownika podczas wykonywania czterech zadań. Porównanie obu wariantów przeprowadzono w kontrolowanych warunkach, zachowując jednocześnie identyczną strukturę zadań, układu i celów do wykonania. Rożnicę pomiędzy A i B  ograniczono do badanych zmiennych, aby następnie zestawić zachowania użytkowników i ocenić wpływ decyzji projektowych na przebieg rozgrywki 
 
 Zakres badanych obszarów został wyselekcjonowany na podstawie wcześniejszych wyników analizy heurystycznej oraz badania ankietowego. Do badań włączono te aspekty interakcji, które okazały się problematyczne lub niejednoznaczne w obu etapach analizy. Jednym z najczęściej wskazywanych problemów w doświadczeniach VR były mechanizmy lokomocji. Informacja zwrotna, mimo że nie była wskazywana przez użytkowników jako obszar krytyczny, została uwzględniona w badaniu ze względu na wyniki analizy heurystycznej. W większości analizowanych gier VR była ona ograniczona lub pomijana, dlatego zdecydowano się na jej uwzględnienie w badaniu końcowym w celu oceny wpływu intensywności informacji zwrotnej na jakość rozgrywki i przebieg interakcji.
 
 Z zakresu badań świadomie zrezygnowano z mechaniki ekwipunku, pomimo problemów wykrytych w pierwszym etapie analizy. Decyzja ta wynikała z faktu, że zidentyfikowane trudności dotyczyły wyłącznie jednego typu rozwiązania, tj. ekwipunku realizowanego w formie panelu osadzonego w przestrzeni. Pozostałe warianty uzyskały dobre oceny użyteczności i były wskazywane w badaniu ankietowym jako rozwiązania udane. Uznano zatem, że uwzględnienie mechaniki ekwipunku nie wniesie istotnej wartości do dalszych badań, a zakres pracy zawężono do obszarów generujących największe i najczęściej występujące problemy użytkowe. Dodatkowo obszar ten pokrywał się częściowo funkcjonalnie z interfejsem panelowym, który został już objęty zakresem badania.

 
\subsection{Metodologia badań}

W celu wyeliminowania efektu uczenia się badanie zostało zaprojektowane zgodnie z modelem międzygrupowym. Uczestnicy zostali podzieleni na dwie niezależne grupy badawcze, a każda z nich testowała wyłącznie jeden wariant interfejsu. Wybór projektu międzygrupowego został podyktowany obawą, że schemat wewnątrzgrupowy mógłby prowadzić do zapamiętania schematu zadania przez użytkowników i zakłócenia wyników. Obie grupy zostały podzielone pod względem poziomu doświadczenia z technologią VR tak aby grupy były zrównoważone. Każda z nich zawierała 3 użytkowników początkujących i 2  zaawansowanych. Zapewniło to kontrole nad możliwym rożnym poziomem doświadczenia osób badanych i pozwoliło na sprawdzenie czy wcześniejsze doświadczenie wpłynie na odbiór danego wariantu. W badaniu wzieło udział 10 dorosłych osób. Rekrutacja odbywała się 

Kto wziął udział w badaniu? (dobór próby)
Jaką formę badania wybrałeś – stacjonarną czy online?
Jak przebiegało badanie – kiedy, gdzie i jak długo trwało?
Jakie zmienne były kontrolowane lub eliminowane, aby wyniki były rzetelne?
Jak zapewniono anonimowość i zgodność z zasadami etycznymi?
\subsection{Wyniki badań A/B}
\subsection{Interpretacja wyników}
\subsection{Podsumowanie pracy}



