\chapter{Projektowanie interfejsu VR}


\section{Analiza rozwiązań istniejących już na rynku}
Projektowanie interfejsu do VR może odbywać się na kilka różnych sposobów różniących się od siebie sposobem integracji z wirtualnym środowiskiem i tym jak prezentowane są istotne informacje. Aby odpowiednio przygotować się do projektowania, należy zapoznać się z najczęściej stosowanymi rozwiązaniami w istniejącymi już na rynku produktach. Należy zrozumieć co twórcy gier i symulatorów uznali za ważniejsze, pełną immersje czy raczej stawiali na tradycyjne menu i lewitujące w przestrzeni wirtualnej panele. W pierwszej kolejności warto jednak poznać kilka podstawowych rodzajów interfejsow i czym sie charakteryzują. Pozwoli to na lepsze zrozumienie jak i kiedy są one wykorzystywane. Na potrzeby analizy zastosowano podział interfejsów na cztery najczęściej stosowane rodzaje. Interfejs jako część świata gry, Klasyczny panel menu (Non-diegetic UI), Interfejs panelowy w środowisku VR (Spatial UI) oraz HUD (Head-Up Display.
Pierwszy rodzaj polega na wpleceniu interfejsu w elementy już istniejące w świecie wirtualnym, użytkownik żeby go zobaczyć musi fizycznie skierować głowę na dany element. Drugi z kolei wykorzystuje tradycyjny panel menu, ktory możemy spotkać na stronach internetowych czy w grach na komputer. Elementy interfejsu nie są częścią otoczenia a sam panel jest nakładką na widok użytkownika. Kolejne podejście to interfejs panelowy ale umieszczony w środowisku wirtualnym. Typ ten charakteryzuje się unoszącym się w powietrzu menu, które może być sterowane za pomocą kontrolera lub ruchami rąk. Ostatni typ to HUD, rodzaj stosowany często w tradycyjnych grach. Najważniejsze informacje są "przyklejone" do widoku gracza, pozostając cały czas w jego polu widzenia niezależnie od miejsca w którym znajduje się użytkownik w wirtualnym otoczeniu.

\subsection{Half-Life: Alyx}

Pierwszy rodzaj polega na wpleceniu interfejsu w elementy już istniejące w świecie wirtualnym a użytkownik żeby go zobaczyć musi fizycznie skierować głowe na dany element. Może to być wyświetlacz kokpitu  czy ekran Monitora
No Man’s Sky VR, Interfejs panelowy w środowisku VR (Spatial UI)


(Jonathan Linowes w Unity Virtual Reality Projects)



\subsection{Beat Saber}
Phasmophobia VR
Beat Saber

\subsection{No Man's Sky}

https://ekspert.ceneo.pl/najlepsze-gry-vr


\subsection{Symulator VR „Zaawansowane procedury medyczne”
}
"Rozwija praktyczne umiejętności studentów w zakresie segregacji medycznej, udzielania kwalifikowanej pierwszej pomocy, podstawowej pierwszej pomocy oraz ratownictwa medycznego. Posiada edytor umożliwiający wybór odpowiedniego środowiska, konfigurację pacjentów, dostępnego sprzętu i wartości referencyjnych. Dostępny jest w trybach: egzaminacyjnym lub ćwiczeniowym; jedno bądź wieloosobowym oraz w wariancie PC i VR."



\subsection{Komentarz}
Zastanawiam się czy nie skupić się głownie na symulatorach. Interfejsy w grach najczęściej są wyjątkowo dopracowane aby rozgrywka odbywała się płynnie. W przypadku symulatorów szczególnie edukacyjnych rozwiązania są rozbudowane pod względem mechanik w samej rozgrywce ale sposób przekazywania informacji jest tragiczny, Interakcje odbywają sie głownie za pomocą lasera a sam interfejs utrudnia wykonywanie czynności. Brakuje tam naśladowania jak te czynnści odbywają się w prawdziwym świecie. Przykładowo w symulatorze kryminalistyki możemy pobierać odciski buta itp ale wyciaganie przedmiotow odbywa sie nieintuicyjnie i rozpraszająco za pomocą interfejsu panelowego umieszczonego w przestrzeni. Zaprojektowanie interfejsu głownie do symulatorów będzie również bardziej logiczne bo głownym ich celem jest nauka a w przypadku gier cele mogą być rozne. Niektore chcą wystraszyc gracza a inne żeby spokojnie eksplorował stworzony świat.

https://uxdesign.cc/vr-diegetic-interfaces-dont-break-the-experience-554f210b6e46


\section{Tworzenie prototypów interfejsów użytkownika dla VR}
Aby uniknąć wielokrotnej implementacji niezliczonej ilości ilości interfejsów 
\section{Testowanie różnych układów elementów i metod interakcji}
%(kontrolery, gesty, głos).
\section{Projektowanie interaktywnych obiektów}
\subsection{obiekt fizyczny dynamiczny}
\subsection{obiekt fizyczny statyczny}
\subsection{obiekt wirtualny}
\section{Tworzenie elementów wizualnych}
%Tworzenie atrakcyjnych wizualnie i funkcjonalnych przycisków.


