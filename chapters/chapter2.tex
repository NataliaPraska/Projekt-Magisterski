\chapter{Podstawowe Zasady UX/UI designu}

\section{Wprowadzenie do UX Designu}

 UX design, czyli projektowanie doświadczeń użytkownika, to proces tworzenia zapewniający pozytywne doświadczenia użytkownikom poprzez optymalizacje interakcji z daną stroną, systemem czy produktem. Początkowo zagadnienie to było kojarzone głównie z projektowaniem stron internetowych i aplikacji, jednak współczesne podejście znajduje zastosowanie w każdej dziedzinie. Obecnie UX design obejmuje projektowanie przeróżnych produktów, od stron internetowych po urządzenia codziennego użytku jak urządzenia AGD, samochody czy nawet meble. Aktualnie UX odnosi się nie tylko do wirtualnych interfejsów ale także do interakcji ze zwykłymi przedmiotami z którymi styczność mamy wszyscy.


\section{Podstawowe Zasady UX/UI}

Autorzy rożnych książek wyróżniają szereg fundamentalnych zasad UX designu, które pomagają w tworzeniu bardziej efektywnych, intuicyjnych i przyjemnych doświadczeń użytkowników.
Wśród kluczowych zasad projektowania UX/UI wyróżnia się:


\subsection{Ergonomia i Prostota}
Produkt powinien być projektowany tak, by był wygodny i łatwy w użyciu. Dobre projektowanie UX uwzględnia nie tylko interakcje użytkownika z urządzeniem, ale również sposób, w jaki przetwarza on informacje wyświetlane na ekranie. Produkt powinien być dostosowany do naturalnych zdolności i ograniczeń użytkownikow zapewniając im płynne i efektywne osiąganie zamierzonych celow.(Don Norman, 2013).

Zasada Prostoty oznacza elimniowanie zbędnych elementow i skomplikowanych funkcji, ktore mogą rozpraszać uwage odbiorcy lub sprawiać że produkt staje sie trudny w obsłudze. Koncentruje sie głownie na tym z czego użytkownik będzie rzeczywiscie korzystał Prostota podczas projektowania interfejsow polega na tym, ze wszystkie funkcjonalności są dostępne dla użytkownika w oczywisty i jak najprostszy sposob. Dązenie do prostoty w projekcie jest istotne ponieważ pomaga zapobiec przeciązeniu poznawczemu ktore może wystąpić gdy system jest zbyt skomplikowany. 


\subsection{Widoczność i Intuicyjność}

Dobrze zaprojektowany pod względem widoczności interfejs powinien sprawiać by użytkownik był w stanie łatwo dostrzec dostępne opcje funkcje bez konieczności szukania. Elementy powinny by umieszczone w miejscach w których naturalnie się ich spodziewają. Pozwala to na szybkie zrozumienie sposobu poruszania się w systemie. Zasada ta uwzględnia również umiejscowienie mniej istotnych elementów takie jak ustawienia lub dodatkowe funkcje. Nie muszą one być widoczne na pierwszy rzut oka ale ważne jest aby były one nadal łatwe do zlokalizowania w razie potrzeby. Don Norman

Intuicyjność oznacza ze interfejs powinien być zaprojektowany w taki sposób by  użytkownik mógł naturalnie zrozumieć jego działanie opierając się na swoich wcześniejszych doświadczeniach i oczekiwaniach. Elementy powinny być widoczne i łatwe do zlokalizowania oraz reagować zgodnie z jego przyzwyczajeniami. Sposób poruszania się po systemie powinien być dla nich oczywisty. Steve Krug

\subsection{Dostępność i Informacja zwrotna}
Zasada dostępności jest bardzo ważna aby zapewnić by wszyscy użytkownicy, niezależnie zdolności fizycznych, sensorycznych czy poznawczych  byli w stanie korzystać z aplikacji. Treść interfejsu powinna być jasna i zrozumiała a teksty powinny być odpowiednio widoczne  z dobrym kontrastem miedzy tłem a literami. Elementy interaktywne takie jak przyciski suwaki powinny być łatwe do znalezienia i i użyteczne. Powinien również  oferować użytkownikom dostosowania jego wyglądu , na przykład zmianę rozmiaru tekstu lub kontrastu. Designing Interfaces Jenifer Tidwell

Jest to reakcja systemu na działania które wykonuje użytkownik pozwalając mu na kontrolowanie swoich interakcji z systemem i ich efektów w czasie rzeczywistym. Dobre projektowanie interfejsu zapewnia, że każda interakcja użytkownika, taka jak kliknięcie przycisku czy zmiana ustawienia, zostanie natychmiastowo potwierdzona odpowiednią reakcją systemu. Reakcja może przybierać różne formy jak wizualna czy dźwiękowa.(Saffer, 2006)



\section{Róznice w projektowaniu interfejsu VR}
%OPC wzorce projektowe
\section{Zasady  projektowania UX/UI w kontekście aplikacji VR}


W kontekście VR projektowanie interfejsów staje się szczególnie problematyczne ze względu na konieczność osiągnięcia równowagi między imersją w wirtualnej rzeczywistości a zapewnieniem intuicyjnej interakcji oraz komfortu użytkowania dlatego istotne jest zastosowanie sprawdzonych zasad UX/UI, które umożliwią stworzenie rozwiązania dla specyfiki tej technologii.