\chapter{Wprowadzenie do tematyki pracy}
Wirtualna i Rzeczywistość jest technologią, która wraz z pojawieniem się pierwszych konsumenckich headsetów VR przeszła w nowy etap rozwoju, wykraczający poza wcześniejsze, ograniczone zastosowania w środowiskach specjalistycznych i badawczych. Od tego momentu zaczeła ona wzbudzać zainteresowanie nie tylko w środowisku naukowym ale również w komercyjnym. Na rynku gier ta technologia otworzyła nowe możliwości, zarówno w tworzeniu nowych tytułów jak i przenoszeniu już istniejących do środowiska VR. Zainteresowanie technologią wzrastało wraz z pojawieniem się nowych urządzeń lub gier, jednak gracze często rezygnowali z dalszego korzystania z gier VR z powody problemów zdrowotnych takich jak choroby lokomocyjnej czy dyskomfortu, a także z uwagi ma niską jakość oferowanych produktów oraz ograniczoną liczbę dopracowanych tytułów. Tym samym ogromny potencjał technologii, umożliwiającej tworzenie immersyjnych trójwymiarowych środowisk niedostępnych w rzeczywistym świecie, nie zawsze przekładał się na pozytywny odbiór końcowych produktów przez graczy 
Jednym z większych problemów nadal jest słaba jakość doświadczeń, z którymi użytkownik styka się już na początkowym etapie kontaktu z grą. W wielu przypadkach pierwsze minuty gry wiążą się z dezorientacją, koniecznością przyswojenia nieintuicyjnych schematów sterowania oraz nadmiarem informacji prezentowanych w sposób sprzeczny z oczekiwaniami odbiorcy. Używane w grach rozwiązania projektowe związane UI i mechanizmami interakcji są często niespójne, mało intuicyjne lub skopiowane z bezpośrednio z gier 2D i 3D. W efekcie negatywnie wpływają one na odbiór gry oraz ograniczają poziom zanurzenia użytkownika w wirtualnym świecie.
Brak ujednoliconych i zweryfikowanych w srodowisku testowym wytycznych projektowych dedykowanych grom VR skutkuje, że jakość końcowego produktu jest ściśle powiązana z wiedzą zespołu dewloperskiego w zakresie projektowania doświadczeń użytkownika.


\section{Kontekst rozwoju technologii VR}
Aby w pelni zrozumieć obecną sytuacje na rynku VR, należy najpierw zrozumieć jak  zestawy zmieniły się na przestrzeni lat. Podczas pierwszej fala zainteresowania miała miejsce w 2016 roku kiedy na rynek wyszły pierwsze konsumenckie zestawy VR. Wydarzenie to stanowiło istotny krok w kierunku upowszechnienia i rozwoju tej technologii. Urządzenia te sprawiły, że możliwe stało się doświadczenie VR w warunkach domowych. Pierwsze headsety miały niestety sporo ograniczeń, wymagały one przewodowego połączenia z wydajnym komputerem oraz zastosowania zewnętrznych sensorów śledzących ruch. Konieczne było wydzielonie miejsca na sensory oraz wykonanie mapowania przestrzeni. Przemieszczenie mebli lub sensorów oznaczało ponowną kalibracje.  Dodatkowo brak obsługi wielu profili utrudniał dzielenie się urządzeniem w warunkach domowych. 
W kolejnych latach zrezygnowano z wielu problematycznych rozwiązań stosowanych w pierwszych wersjach urządzeń. W nowych generacjach zrezygnowano z zewnętrznych sensorów i zastosowano śledzenie ruchu inside-out czyli opartego na kamerach wbudowanych bezpośrednio w kask. Proces konfiguracji początkowej również uległ zmianie i mapowanie pokoju jest wykonywane automatycznie z funkcją wykrywania przeszkód jak meble czy ściany.  Większość urządzeń jest aktualnie bezprzewodowa i została wyposażona w  wbudowane w okulary VR  zasilanie, ponadto dodano funkcje szybkiego przełączania profili użytkowników.
Razem z rozwojem urządzeń nastąpił rozwój narzędzi symulacyjnych oraz środowisk testowych wspierających projektowanie i sprawdzanie aplikacji VR bez fizycznego posiadania googli. Umożliwiało to symulacje ruchu interakcji i zachowań użytkownika w wirtualnym środowisku. Branża VR dążyła  do ujednolicenia rozwiązań technicznych i stworzyła wspólne standardy API i narzedzia deweloperskie które miały na celu ułatwić tworzenie spójnych gier i aplikacji na różne platformy. Standaryzacja ta odnosiła się jednak przede wszystkim do warstwy technicznej, obejmującej obsługę urządzeń, kontrolerów oraz systemów śledzenia ruchu. W obszarze projektowania interfesów użytkownika oraz interakcji można zauważyć dużą róznorodność podejść projektowych. Wynika to z przyjętych konwncji projektowych oraz z narzędzi oferowanych przez różne silniki jak Unity czy Unreal Engine. W rezultacie jakość doświadczeń użytkownika różni się pomiędzy poszczególnymi produkcjami, co ma bezpośredni wpływ na odbiór technologii VR przez konsumentów.
Rozwój i ujednolicenie sprzętu VR oraz narzędzi programistycznych, a także ich integracją z silnikami takimi jak Unity i Unreal Engine, wyraźnie obniżył próg wejścia dla twórców. Dodatkowym ułatwieniem stała się również większa dostępność narzędzi symulacyjnych oraz trybów emulacji wirtualnego środowiska, które umożliwiają projektowanie i testowanie aplikacji bez konieczności stałego fizycznego posiadania zestawu VR. Okres ten początkowo charakteryzował się dynamicznym wzrostem zainteresowania technologią jednak wraz z jej upowszechnieniem użytkownicy zaczeli zauważać liczne nieintuicyjne i niekomfortowe rozwiązania projektowe. W ostatnich latach zainteresowanie VR wśród konsumentów uległo osłabieniu
Według danych opublikowanych przez firmę analityczną -badawczą Omdia, w 2024 roku sprzedaż konsumenckich zestawów VR spadła o około 10 procent, z poziomu 7,7 mln do 6,9 mln sprzedanych egzemplarzy. Dla porównania, w okresie pierwszej fali popularyzacji technologii VR, około roku 2016, sprzedano prawie  12 mln urządzeń. Prognozy wskazują, że spadkowy trend może utrzymać się co najmniej do końca 2025 roku (Omdia, 2024).

	\begin{figure}[!htb]
    \centering
    \includegraphics[width=1\textwidth]{images/1.png}
    \caption{Sprzedaż konsumenckich zestawów wirtualnej rzeczywistości i wydatki na treści VR, (2024); źródło: Omedia}
    \end{figure}

Jednym z istotnych powodów takiej sytuacji jest deficyt atrakcyjnych i nowatorskich gier oraz aplikacji VR, a także ograniczona liczba produkcji dopracowanych nie tylko pod względem zawartości, takiej jak fabuła czy mechaniki rozgrywki, lecz również pod kątem interfejsu użytkownika, sposobów interakcji oraz ogólnego komfortu użytkowania. Nowe produkcje często cechuje powtarzalność i przeciążenie użytkownika już na początku rozgrywki, co negatywnie wpływa na komfort korzystania z VR. Pomimo obecnych trudności prognozy Omdia na kolejne lata pozostają umiarkowanie optymistyczne i wskazują, że rok 2026 może stanowić początek ponownego wzrostu rynku VR. Z tego względu obecny etap rozwoju technologii można uznać za szczególnie istotny z perspektywy projektowania gier VR, ponieważ wraz z ponownym wzrostem zainteresowania użytkowników kluczowe znaczenie będą miały jakość interfejsów użytkownika oraz spójność zastosowanych mechanizmów interakcji.



\section{Znaczenie interfejsu użytkownika i interakcji w VR}

W wirtualnej rzeczywistości świat, w którym użytkownik przebywa, jest istotny, jednak kluczowe znaczenie ma to, jakie działania może on w tym świecie podjąć oraz w jaki sposób może na niego oddziaływać. Interfejs użytkownika oraz mechanizmy interakcji należą do podstawowych elementów umożliwiających budowanie wrażenia obecności. W sytuacji, gdy którykolwiek z tych elementów jest niezgodny z oczekiwaniami użytkownika lub naturalnymi reakcjami ciała, doświadczenie przestaje być komfortowe, a immersja zostaje zakłócona.
Don Norman wskazuje, że dobrze zaprojektowany produkt nie powinien wymagać dodatkowych wyjaśnień, lecz opierać się na znanych użytkownikowi schematach działania. W kontekście wirtualnej rzeczywistości oznacza to projektowanie interakcji w sposób możliwie zbliżony do naturalnych ludzkich odruchów. Przykładowo, otwarcie drzwi powinno polegać na wykonaniu ruchu ręką odpowiadającego rzeczywistemu gestowi, a nie na użyciu abstrakcyjnego przycisku na kontrolerze lub wyborze opcji z menu. Rozwiązania odbiegające od takich wzorców wymagają od użytkownika dodatkowego wysiłku poznawczego i wydłużają proces adaptacji do środowiska VR.
Im większa liczba niespójności pomiędzy sposobem interakcji a oczekiwaniami użytkownika, tym większe jest ryzyko wystąpienia frustracji, niepokoju oraz zmęczenia poznawczego. W skrajnych przypadkach niewłaściwie zaprojektowane interakcje mogą również przyczyniać się do występowania objawów choroby lokomocyjnej, co znacząco obniża chęć dalszego korzystania z aplikacji lub gry.
Z tego powodu sam interfejs użytkownika nie jest wystarczający do stworzenia pozytywnego doświadczenia. Kluczową rolę odgrywa całokształt doświadczeń użytkownika (User Experience), obejmujący zarówno aspekty funkcjonalne, jak i subiektywne odczucia pojawiające się podczas korzystania z aplikacji. Interfejs użytkownika w VR stanowi warstwę pośredniczącą pomiędzy użytkownikiem a systemem i odpowiada za przekazywanie informacji, sygnalizowanie stanu gry oraz umożliwienie wykonywania działań.
Sposób zaprojektowania tej warstwy ma bezpośredni wpływ na to, w jaki sposób użytkownik interpretuje dostępne możliwości interakcji oraz jak postrzega spójność i logikę świata przedstawionego. Oznacza to, że ocena jakości interfejsu i interakcji nie może być dokonywana w oderwaniu od kontekstu, w jakim są one wykorzystywane, lecz powinna uwzględniać charakter rozgrywki oraz zadania stawiane przed użytkownikiem.
Należy jednocześnie podkreślić, że skuteczność interfejsu użytkownika oraz zastosowanych mechanizmów interakcji w VR zależy od kontekstu projektowego, w tym od typu gry oraz charakteru zadań wykonywanych przez użytkownika. Różnice pomiędzy gatunkami oraz tempem rozgrywki sprawiają, że te same rozwiązania interfejsowe mogą w odmienny sposób wpływać na poziom immersji i komfort użytkowania. Uwzględnienie specyfiki scenariuszy rozgrywki w środowisku wirtualnej rzeczywistości sprzyja projektowaniu interfejsów użytkownika oraz mechanizmów interakcji lepiej dopasowanych do danego typu doświadczenia.


\section{Wyzwania projektowania UI w VR}

Projektowanie interfejsów użytkownika w VR nie polega na prostym przeniesieniu zasad stosowanych w grach 2D lub klasycznych grach 3D. Środowisko trójwymiarowe wprowadza nowe wyzwania, takie jak silne powiązanie interfejsu z ruchem ciała użytkownika, podatność na dezorientację przestrzenną oraz ograniczenia percepcyjne wynikające z konstrukcji zestawów VR.
Badania z zakresu projektowania interfejsów w środowisku wirtualnej rzeczywistości wykazują, że jednym z największych wyzwań dla projektantów VR pozostaje zachowanie ergonomii interakcji, spójności pomiędzy różnymi grami oraz utrzymanie immersji przez cały czas korzystania z produktu. Systematyczny przegląd literatury przeprowadzony przez Lima, Catapan i Zeredo (2024)) podkreśla, że projektowanie interfejsów VR wciąż napotyka istotne trudności, szczególnie w obszarze ergonomii fizycznej i poznawczej. Podobne wnioski przedstawiają García, Cano i Moreira (2022), wskazując, że jakość doświadczenia użytkownika w VR jest silnie uzależniona od sposobu zaprojektowania interfejsu oraz mechanizmów interakcji. Niewłaściwe rozmieszczenie elementów interfejsu, nadmierna liczba bodźców wizualnych, brak czytelnych i spójnych wzorców sterowania lub nieergonomiczne interakcje mogą znacząco obniżyć poziom immersji, komfort użytkowania oraz utrudniać odbiór i ocenę aplikacji VR.
Dodatkowym problemem jest konieczność pogodzenia czytelności interfejsu z zachowaniem immersji. Klasyczne rozwiązania typu HUD mogą zaburzać iluzję obecności, natomiast interfejsy diegetyczne, choć bardziej immersyjne, bywają mniej czytelne i trudniejsze w obsłudze. Projektowanie UI w VR wymaga zatem świadomych kompromisów oraz doboru rozwiązań adekwatnych do kontekstu aplikacji.


\section{Luki badawcze i motywacja podjęcia tematu}
Pomimo ciągłego doskonalenia technologii VR proces projektowy gier w małych i średnich zespołach deweloperskich nadal opiera się w głównej mierze na doświadczeniu i intuicji zespołu. Większość projektów jest realizowana bez przeprowadzenia badań wstępnych z grupą docelową czy testów użyteczności gotowych rozwiązań. Często stosowaną praktyką jest wzorowanie się na rozwiązaniach wdrażanych przez podobne lub po prostu popularne produkcje i wklejanie ich do rozgrywki. Zdarza się również że projektanci po ogólnodostępne wytyczne znalezione w internecie, takie jak dokumentacje deweloperskie platform VR, które prezentują ogólne praktyki nie wskazując ich ograniczeń kontekstowych.
Rozwiązania uniwersalne pomijają wiele istotnych różnic między gatunkami gier, jako przykład warto się zastanowić nad grami sumlatorowymi i RPG akcji. Pierwszy gatunek czyli takie gry jak symulatory lotu lub jazdy wymagają realistycznych i opartych na prawdziwym świecie interakcji. Wszystkie sprzeczności takie jak brak możliwości chwycenia dźwigni czy naciśnięcia przycisku mogą prowadzić do obniżenia imersji oraz utrudnień w rozgrywce odbiorcy. Wprawdzie istnieją opracowania książkowe i teoretyczne które porównują interfejsy i interakcje VR, jednak brakuje empirycznych badań porównawczych które sprawdzałyby skuteczność rozwiązań w realnej grze lub środowisku testowym z udziałem użytkowników końcowych. Luka ta stanowiła dla mnie główną motywacje podjęcia tego tematu pracy.
Celem niniejszej pracy jest analiza wplywu interfejsu użytkwonika oraz mechanizmów interakcji na poziom immersji w grze VR reprezentującej wybrany typ gatunkowy. W ramach tej pracy zaprojektowane i zaimplementowane zostanie środowisko testowe wraz z podstawowymi interakcjami dla tego typu rozgrywki. Następnie porównane zostaną dwa odmienne podejścia do projektowania interfejsu. Ocena skuteczności zastosowanych rozwiązań zostanie przeprowadzona na podstawie badań z udziałem użytkowników, obejmujących m.in. pomiar immersji, obserwację zachowań oraz subiektywną ocenę komfortu i intuicyjności interakcji. Otrzymane wyniki pozwolą określić, które rozwiązania interfejsowe i interakcyjne w mniejszym stopniu zaburzają immersję i lepiej wspierają doświadczenie użytkownika w kontekście analizowanego typu gry VR.
