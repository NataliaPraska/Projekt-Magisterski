\chapter{Wprowadzenie do tematyki pracy}

Wirtualna rzeczywistość stanowi technologię, która wraz z pojawieniem się pierwszych konsumenckich zestawów gogli VR wkroczyła w nowy etap rozwoju, wykraczający poza dotychczasowe, silnie ograniczone zastosowania w wyspecjalizowanych środowiskach przemysłowych i badawczych. Od tego momentu zaczęła ona wzbudzać zainteresowanie nie tylko w środowisku naukowym, ale również w komercyjnym. Na rynku gier ta technologia otworzyła nowe możliwości, zarówno w tworzeniu nowych tytułów, jak i przenoszeniu już istniejących do środowiska VR. Zainteresowanie technologią wzrastało wraz z pojawieniem się nowych gier i urządzeń. Gracze często przestawali korzystać z gier VR z powodu dolegliwości zdrowotnych, takich jak choroba lokomocyjna czy ogólny dyskomfort, ale również ze względu na niską jakość dostępnych produkcji oraz niewielką liczbę starannie dopracowanych tytułów. Tym samym ogromny potencjał technologii, umożliwiającej tworzenie immersyjnych trójwymiarowych środowisk niedostępnych w rzeczywistym świecie, nie zawsze przekładał się na pozytywny odbiór końcowych produktów przez graczy. 
Jednym z większych problemów nadal była słaba jakość doświadczeń, z którymi użytkownik styka się już na początkowym etapie kontaktu z grą. W wielu przypadkach pierwsze minuty gry wiążą się z dezorientacją, koniecznością przyswojenia nieintuicyjnych schematów sterowania oraz nadmiarem informacji prezentowanych w sposób sprzeczny z oczekiwaniami odbiorcy. Używane w grach rozwiązania projektowe związane z UI i mechanizmami interakcji są często niespójne, mało intuicyjne lub skopiowane bezpośrednio z gier 2D. W efekcie negatywnie wpływa to na odbiór gry i ogranicza poziom zanurzenia użytkownika w wirtualnym świecie.

Pomimo istnienia ogólnych wytycznych producentów sprzętu oraz pojedynczych badań branżowych, nadal brakuje ujednoliconych i zweryfikowanych instrukcji projektowych dedykowanych konkretnym gatunkom gier VR. W konsekwencji jakość końcowego produktu pozostaje ściśle uzależniona od doświadczenia zespołu deweloperskiego w zakresie projektowania interakcji i doświadczeń użytkownika.

\section{Kontekst rozwoju technologii VR}
Aby w pełni zrozumieć obecną sytuację na rynku VR, należy prześledzić zmiany, jakie zaszły w konstrukcji i sposobie użytkowania zestawów na przestrzeni lat. Pierwsza fala zainteresowania miała miejsce w 2016 roku, kiedy na rynek weszły pierwsze konsumenckie zestawy VR. Wydarzenie to stanowiło istotny krok w kierunku upowszechnienia i rozwoju tej technologii. Urządzenia te sprawiły, że możliwe stało się doświadczenie VR w warunkach domowych.

Pierwsze headsety miały jednak liczne ograniczenia sprzętowe i konfiguracyjne. Obejmowały one m.in. konieczność przewodowego połączenia z wydajnym komputerem oraz stosowanie zewnętrznych sensorów śledzących ruch. Konieczne było także wydzielenie odpowiedniej przestrzeni na sensory i wykonanie mapowania otoczenia, co stanowiło utrudnienie, szczególnie dla użytkowników niedoświadczonych, którzy nie mieli wcześniejszego kontaktu z technologią VR. Przemieszczenie mebli lub sensorów oznaczało ponowną kalibrację. Dodatkowo brak obsługi wielu profili utrudniał dzielenie się urządzeniem w warunkach domowych.

W kolejnych latach zrezygnowano z wielu problematycznych rozwiązań stosowanych w pierwszych wersjach urządzeń. W nowych generacjach zrezygnowano z zewnętrznych sensorów i zastosowano śledzenie ruchu \textit{inside-out}, oparte na kamerach wbudowanych bezpośrednio w kask. Proces konfiguracji początkowej również uległ zmianie, a mapowanie pokoju jest wykonywane automatycznie z funkcją wykrywania przeszkód, takich jak meble czy ściany. Większość urządzeń jest aktualnie bezprzewodowa i została wyposażona w wbudowane w okulary VR zasilanie. Ponadto dodano funkcje szybkiego przełączania profili użytkowników.

Wraz z rozwojem technologii VR pojawiły się narzędzia umożliwiające projektowanie i testowanie gier oraz aplikacji VR bez konieczności fizycznego posiadania zestawu. Rozwiązania te obniżyły próg wejścia dla mniejszych zespołów i niezależnych twórców, umożliwiając im tworzenie oraz weryfikację podstawowych mechanik i interakcji na wcześniejszych etapach procesu projektowego.

Branża dążyła do ujednolicenia rozwiązań technicznych i stworzyła wspólne standardy API oraz narzędzia deweloperskie, które miały na celu ułatwienie tworzenia spójnych aplikacji na różne platformy. Standaryzacja ta odnosiła się jednak przede wszystkim do warstwy technicznej, obejmującej obsługę urządzeń, kontrolerów oraz systemów śledzenia ruchu.

W niewielkim stopniu dotyczyła natomiast projektowania interfejsów użytkownika i mechanizmów interakcji, pozostawiając w tym obszarze znaczną swobodę interpretacyjną twórcom. Wynika to m.in. z przyjętych konwencji projektowych oraz z narzędzi oferowanych przez silniki do tworzenia aplikacji 2D i 3D, takie jak \textit{Unity} czy \textit{Unreal Engine}. W rezultacie jakość doświadczeń użytkownika różni się pomiędzy poszczególnymi produkcjami, co ma bezpośredni wpływ na odbiór technologii VR przez konsumentów.

Okres ten początkowo charakteryzował się dynamicznym wzrostem zainteresowania technologią, jednak wraz z jej rozpowszechnieniem użytkownicy zaczęli zauważać liczne nieintuicyjne i niekomfortowe rozwiązania projektowe. W ostatnich latach zainteresowanie VR wśród konsumentów uległo osłabieniu, a analitycy przewidywali pesymistyczne prognozy dla tej technologii \cite{UX-w-XR}.

Według danych opublikowanych przez firmę analityczną-badawczą \textit{Omdia}, w 2024 roku sprzedaż konsumenckich zestawów VR spadła o około 10 procent, z poziomu 7,7 mln do 6,9 mln sprzedanych egzemplarzy. Dla porównania, w okresie pierwszej fali popularyzacji technologii VR, około roku 2016, sprzedano prawie  12 mln urządzeń. Podane Prognozy wskazują, że spadkowy trend może utrzymać się co najmniej do końca 2025 roku \cite{Prognozy}.

	\begin{figure}[!htb]
    \centering
    \includegraphics[width=1\textwidth]{images/1.png}
    \caption{Sprzedaż konsumenckich zestawów wirtualnej rzeczywistości i wydatki na treści VR, (2024); źródło: Omedia}
    \end{figure}

Jednym z istotnych powodów obserwowanych trudności na rynku jest deficyt atrakcyjnych i nowatorskich gier oraz aplikacji VR. Problemem pozostaje także ograniczona liczba produkcji dopracowanych nie tylko pod względem zawartości (np. fabuły czy mechanik rozgrywki), lecz również pod kątem interfejsu użytkownika, sposobów interakcji oraz ogólnego komfortu użytkowania.

Nowe produkcje często charakteryzują się powtarzalnością oraz przeciążeniem użytkownika nadmiarem bodźców i informacji już na początku rozgrywki, co obniża komfort korzystania z VR. Skutkuje to szybkim zniechęceniem odbiorców, szczególnie na początkowym etapie kontaktu z grą, i negatywnie wpływa na postrzeganie całej technologii. Pomimo obecnych trudności prognozy Omedia na kolejne lata pozostają umiarkowanie optymistyczne i wskazują, że rok 2026 może stanowić początek ponownego wzrostu rynku VR.

Prognozowany spadek zainteresowania technologią wirtualnej rzeczywistości dobiega końca i zbiega się z zapowiadanymi premierami nowych generacji urządzeń, które w coraz większym stopniu eliminują ograniczenia sprzętowe charakterystyczne dla wcześniejszych zestawów. Wraz z upowszechnieniem bardziej kompaktowych, lżejszych i wygodniejszych rozwiązań maleje wpływ samego sprzętu na komfort użytkowania. W konsekwencji czynniki technologiczne przestają być głównym źródłem ograniczeń doświadczenia użytkownika. Odpowiedzialność za jakość interakcji oraz poziom immersji w coraz większym stopniu przenosi się na warstwę projektową gry. Obejmuje ona nie tylko fabułę, ale i sposób zaprojektowania interfejsu użytkownika oraz mechanizmów interakcji. To właśnie te elementy decydują obecnie o tym, czy użytkownik jest w stanie swobodnie poruszać się po wirtualnym środowisku i czerpać satysfakcję z rozgrywki.

W tym kontekście uzasadnione staje się podejmowanie badań koncentrujących się na identyfikacji i ocenie rozwiązań interfejsowych oraz mechanizmów interakcji, które mogłyby stanowić wsparcie projektowe również dla mniejszych zespołów. Upowszechnienie wiedzy na temat skutecznych i problematycznych rozwiązań może przyczynić się do poprawy jakości doświadczeń użytkowników oraz zwiększenia różnorodności i konkurencyjności przyszłych produkcji VR.

\section{Znaczenie interfejsu użytkownika i interakcji w VR}

W wirtualnej rzeczywistości świat, w którym użytkownik przebywa, jest bardzo istotny. Kluczowe znaczenie ma jednak to, jakie działania może on w tym świecie podjąć oraz w jaki sposób może na niego oddziaływać. Interfejs użytkownika oraz mechanizmy interakcji należą do podstawowych elementów umożliwiających budowanie wrażenia obecności. W sytuacji, gdy którykolwiek z tych elementów jest niezgodny z oczekiwaniami użytkownika lub naturalnymi reakcjami ciała, doświadczenie przestaje być komfortowe, a immersja zostaje zakłócona.
Don Norman wskazuje, że dobrze zaprojektowany produkt nie powinien wymagać dodatkowych wyjaśnień, lecz opierać się na znanych użytkownikowi schematach działania. W kontekście wirtualnej rzeczywistości oznacza to projektowanie interakcji w sposób możliwie zbliżony do naturalnych ludzkich odruchów i instynktów. Gry wykonane w tej technologii często zapominają o tej zasadzie i zmuszają użytkowników do nauki nowych nieintuicyjnych schematów działania. Przykładowo, otwarcie drzwi powinno polegać na wykonaniu ruchu ręką odpowiadającego rzeczywistemu gestowi, a nie na użyciu abstrakcyjnego przycisku na kontrolerze lub wyborze opcji z menu. Rozwiązania odbiegające od takich rozwiązań wymagają od użytkownika dodatkowego wysiłku poznawczego i wydłużają proces adaptacji do środowiska VR. Im większa liczba niespójności pomiędzy sposobem interakcji a oczekiwaniami użytkownika, tym większe jest ryzyko wystąpienia frustracji, niepokoju oraz zmęczenia poznawczego. W skrajnych przypadkach niewłaściwie zaprojektowane interakcje mogą również przyczyniać się do występowania objawów choroby lokomocyjnej, co znacząco obniża chęć dalszego korzystania z aplikacji lub gry.\cite{Odczucie-gry}

Z tego powodu sam interfejs użytkownika nie jest wystarczający do stworzenia pozytywnego doświadczenia. Kluczową rolę odgrywa całokształt doświadczeń użytkownika (\textit{User Experience}), obejmujący zarówno aspekty funkcjonalne, jak i subiektywne odczucia pojawiające się podczas korzystania z aplikacji.

Interfejs użytkownika w VR stanowi warstwę pośredniczącą pomiędzy użytkownikiem a systemem. Odpowiada za przekazywanie informacji, sygnalizowanie stanu gry oraz umożliwienie wykonywania działań. W tym ujęciu pełni rolę swoistego „tłumacza” pomiędzy logiką systemu a sposobem postrzegania i działania użytkownika. Gdy warstwa ta jest nieczytelna, niespójna lub oparta na nieintuicyjnych konwencjach, komunikacja zostaje zakłócona, co utrudnia zrozumienie dostępnych możliwości.

Ocena jakości interfejsu i interakcji nie może być dokonywana w oderwaniu od kontekstu użycia. Powinna uwzględniać charakter rozgrywki oraz zadania stawiane przed użytkownikiem. Różnice pomiędzy gatunkami i tempem rozgrywki sprawiają, że te same rozwiązania interfejsowe mogą w odmienny sposób wpływać na poziom immersji komfort użytkowania czy efektywność wykonywanych zadań Uwzględnienie specyfiki scenariuszy sprzyja projektowaniu interfejsów użytkownika oraz mechanizmów interakcji lepiej dopasowanych do danego typu doświadczenia.

\section{Wyzwania projektowania UI w VR}

Projektowanie interfejsów użytkownika w VR nie polega na prostym przeniesieniu zasad stosowanych w grach 2D lub stronach internetowych. Środowisko trójwymiarowe wprowadza nowe wyzwania, takie jak silne powiązanie interfejsu z ruchem ciała użytkownika, podatność na dezorientację przestrzenną oraz ograniczenia percepcyjne wynikające z konstrukcji zestawów VR.

Badania z zakresu projektowania interfejsów w środowiskach wirtualnej rzeczywistości wskazują, że jednym z największych wyzwań pozostaje zachowanie ergonomii interakcji, spójności między różnymi grami oraz utrzymanie immersji przez cały czas korzystania z produktu. Systematyczny przegląd literatury przeprowadzony przez Limę, Catapana i Zeredo~\cite{Construstion} podkreśla, że projektowanie interfejsów VR wciąż napotyka istotne trudności, szczególnie w obszarze ergonomii fizycznej i poznawczej. Podobne wnioski przedstawiają García, Cano i Moreira (2022)~, wskazując, że jakość doświadczenia użytkownika w VR jest silnie uzależniona od sposobu zaprojektowania interfejsu oraz mechanizmów interakcji. Niewłaściwe rozmieszczenie elementów interfejsu, nadmierna liczba bodźców wizualnych, brak czytelnych i spójnych sposobów sterowania oraz nieergonomiczne interakcje mogą znacząco obniżyć poziom immersji i komfort użytkowania, a także utrudniać odbiór oraz ocenę aplikacji VR.\cite{UeX}

Dodatkowym problemem jest konieczność pogodzenia czytelności interfejsu z zachowaniem immersji. Klasyczne rozwiązania typu HUD mogą zaburzać iluzję obecności, natomiast interfejsy diegetyczne, choć bardziej immersyjne, bywają mniej czytelne i trudniejsze w obsłudze. Projektowanie UI w VR wymaga zatem świadomych kompromisów oraz doboru rozwiązań adekwatnych do kontekstu aplikacji.

\section{Luki badawcze i motywacja podjęcia tematu}

Pomimo ciągłego doskonalenia technologii VR, proces projektowy gier w małych i średnich zespołach deweloperskich nadal opiera się w głównej mierze na doświadczeniu i intuicji zespołu. Większość projektów jest realizowana bez przeprowadzenia badań wstępnych z grupą docelową czy testów użyteczności gotowych rozwiązań. Często stosowaną praktyką jest wzorowanie się na rozwiązaniach wdrażanych przez podobne lub po prostu popularne produkcje i przeniesienie ich do rozgrywki.

Zdarza się również, że projektanci sięgają po ogólnodostępne wytyczne znalezione w internecie. Należą do nich m.in. dokumentacje deweloperskie platform VR (\textit{Oculus}, \textit{Steam}) oraz niedawne prace badawcze proponujące ujednolicone ramy projektowe \cite{Best-Practices}. Choć źródła te dostarczają cennych, ogólnych wskazówek, często nie uwzględniają specyfiki poszczególnych gatunków gier VR ani ograniczeń kontekstowych opisywanych praktyk.

Rozwiązania uniwersalne pomijają wiele istotnych różnic między gatunkami gier. Dobrym przykładem są gry symulacyjne (np. symulatory lotu lub jazdy). Wymagają one realistycznych odwzorowań działania obiektów i mechanik znanych z rzeczywistości. Każda sprzeczność wobec tych oczekiwań może obniżać immersję i utrudniać rozgrywkę. Gry osadzone w  fikcyjnym świecie nie wymagają takiego podejścia ze ze względu na występowanie elementów, które nie istnieją w prawdziwym świecie. Mogą to być zdolności nadnaturalne, magia lub futurystyczne przedmioty. W takim przypadku realizm przestaje być nadrzędnym kryterium jakości, a projekt wymaga zastosowania odmiennego podejścia i powinien opierać się na wewnętrznej spójności oraz jasno określonych zasadach rządzących wykreowanym światem.

Choć istnieją opracowania książkowe i teoretyczne porównujące interfejsy oraz interakcje VR, wciąż brakuje empirycznych badań porównawczych. Dotyczy to w szczególności prac weryfikujących skuteczność rozwiązań w realnej grze lub w środowisku testowym z udziałem użytkowników końcowych. Luka ta stanowiła główną motywację do podjęcia tematu pracy.
