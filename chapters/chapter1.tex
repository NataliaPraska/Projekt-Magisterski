\chapter{Wprowadzenie do tematyki pracy}


\section{Kontekst rozwoju technologii VR}

W ostatnich latach rozwój wirtualnej rzeczywistości uległ znacznej stagnacji. Według danych podanych przez Omedia(2024)(https://crn.pl/aktualnosci/konsumencki-rynek-vr-spadl-w-2024-r-i-bedzie-spadal-w-2025-r/) wpływy ze sprzedaży zestawów VR  w roku 2024  spadła aż o 10 procent Z 7,7 mln do 6,9 mln sztuk. Kiedy VR weszło po raz pierwszy na rynek  w roku 2016 sprzedane zostało prawie 12 mln sztuk. Aktualny spadek ma się utrzymać aż do końca 2025 roku. Jednym z powodów takiego spadku jest deficyt atrakcyjnych i nowatorskich gier i aplikacji. Nowe produkcje są często niedopracowane i powodują przeciążenie użytkownika przez co chętniej sięgają po prostsze i łatwiej dostępne urządzenia jak konsole czy komputer. Developerzy nie mają pewności czy wejście w rynek VR się im opłaci przez co pojawia się coraz mniej nowych tytułów po które warto sięgnąć a coraz więcej niskobudżetowych powtarzających się produkcji.

Cykliczne spadki od 2016 r. były również spowodowane samym działaniem większości sprzętów oraz ich cenami. Rzeczywistość wirtualna ma jeden z wyższych progów wejścia na rynek w porównaniu z tradycyjnymi platformami jak np. Smartfony czy Konsole. Większość urządzeń ze średniej półki cenowej potrzebuje stałego przewodowego podłączenia do komputera z mocnym procesorem graficznym. Samo posiadanie urządzenia nie było wystarczające do prawidłowego działania wszystkich oferowanych produktów, Wyjątkowo problematyczne okazało się bowiem samo zagospodarowanie odpowiedniej ilości wymaganego miejsca. Starsze wersje wyposażone w specjalne zewnętrzne kamerki wymagały przejścia przez wiele kroków aby poprawnie korzystać z urządzenia co dla osób niedoświadczonych mogło się okazać zbyt skomplikowane. Przeniesienie urządzenia czy zmiana pozycji kamerki zewnętrznej wymagało ponownego skalibrowania i przejścia przez wszystkie etapy. Nawet jeśli udało się wyznaczyć miejsce, ustawić urządzenie, dokonać ręcznego mapowania pokoju i skonfigurować nie zapewniało to komfortowego użytkowania wszystkim domownikom. Nie było możliwości obsługi wielu profilów więc Każda kolejna osoba przed użyciem, musiała ponownie korygować pod siebie ustawienia.

Nowsze generacje urządzeń wprowadziły znaczne usprawnienia działania urządzeń.
Zrezygnowano na przykład z zewnętrznych sensorów, śledzenie odbywa się teraz najczęściej za pomocą wbudowanych w kask kamer. Sama kalibracja startowa urządzenia również uległa znacznemu ułatwieniu. Mapowanie pokoju jest teraz wykonywane automatycznie po maksymalnie minutowym skanowaniu pomieszczenia. Modele wykrywają teraz automatycznie przeszkody takie jak meble czy ściany. Użytkownik nie jest już przywiązany do jednego miejsca, google są teraz częściej bezprzewodowe i wyposażone w dodatkową baterie. Wprowadzenie możliwości zapisywania ustawień dla danego użytkownika i szybkiego przełączania się między profilami ułatwia korzystanie z VR i pozwala na korzystanie większej ilości osób. Mimo tych wszystkich usprawnień nowe urządzenia nadal łamią podstawową zasadę low friction, polegającą na zapewnieniu użytkownikowi możliwości osiągnięcia celu z minimalnym wysiłkiem. Problem ten nie jest aż tak odczuwalny jak w przypadku starszych modeli. Dzięki uproszczeniu najważniejszego etapu jakim jest konfiguracja startowa, znacznie wiecej osób dotrze do fazy rzczeywistej gry. Te usprawnienia sprawiają, że aktualnie największą przeszkodą w Vr jest intuicyjność gier i spójność interakcji. Brak jednoltych wzorców projektowych powoduje, że użytkownik korzystając z każdej nowej produkcji musi ponownie uczyć się sterowania czy nawigacji. Niesie to za sobą konsekwencje i prowadzi do zakłócania immersji, zmęczenia poznawczego i odrzucenia gry lub nawet technologi. 

Mimo obecnych na rynku trudności, prognozy Omedia na najbliższe lata są optymistyczne i szacują, że rok 2026 będzie początkiem wzrostów i odrodzenia technologii. Z tego powodu jest to najlepszy moment na opracowanie spójnych standardów do gier. Gdy nowi i starzy użytkownicy powrócą do urządzeń VR, pierwszym czynnikiem  który zniechęci ich do dalszego użytkowania, będzie interfejs i interakcje. Dostępność informacji zawartych w tej pracy może również wpłynąć pozytywnie na ilość i jakość oferowanych na rynku produktów dostarczając im gotowe do wykorzystania rozwiązania. 

	\begin{figure}[!htb]
    \centering
    \includegraphics[width=1\textwidth]{images/1.png}
    \caption{Sprzedaż konsumenckich zestawów wirtualnej rzeczywistości i wydatki na treści VR, (2024); źródło: Omedia}
    \end{figure}


\section{Znaczenie interfejsu użytkownika i interakcji w VR a rola UX w ich projektowaniu}

W wirtualnej rzeczywistości swiat w którym użytkownik przebywa jest bardzo istotny ale ważniejsze jest to co może z tym światem zrobić, jakie działania podjąć z tym co ma przed sobą. Interfejs i interakcje to jedne z podstawowych elementów które pozwalają na budowanie budowanie wrażenia obecności. Jeśli któryś z nich jest niezgodny z oczekiwaniami ciała użytkownika doświadczenie przestaje być przyjemne a świat immersyjny. Don norman (2013) podkreśla że dobrze stworzony produkt nie powinien wymagać tłumaczenia, powinien opierać się na tym co już znamy. W wirtualnej rzeczywistości polegałoby to na tym żę aby złapać i nacisnąć klamke powinniśmy po prostu zacisnąc dłon a nie kliknąć przcisk x. Otwarcie drzwi powinno odbywać sie za pomocą wstecznego ruchu ręką a nie kliknięciem odpowiedniej opcji dialogowej w menu głównym. Przypadki odbiegające od naturalnych ludzkich odruchów sprawiają ze odbiorca potrzebuje więcej czasu na przystosowanie sie do nowej rzeczywistości. Im więcej niezgodności tym większy może odczuwać niepokój czy frustracje. W rażących sprzecznościach może powodować zmęczenie poznawcze czy objawy choroby lokmocyjnej.

Z tego powodu sam interfejs nie jest wystarczający do stworzenia dobrego produktu, kluczową role pełni doświadzenie użytkownika. Jest to sposób tworzenia produktu skupiający się na odbiorcach, tym jak sie czują podczas użytkowaniana
Interfejs użytkownika w wirtualnej rzeczywistości jest istotnym narzędziem które ma ogromny wpływ na sposób odebrania produktu końcowego i kształtuje całe doświadzczenie. Jest warstwą pośredniczącą między odbiorcą a aplikacją a jego głównym zadaniem jest wymiana informacji. Nie każda gra czy aplikacja może mieć takie same metody interakcji. 


\section{Wyzwania projektowania UI w VR}
\section{Luki badawcze i motywacja podjęcia tematu}
\section{Złożoność procesu projektowego i pełnione role}

