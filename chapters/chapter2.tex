\chapter{Podstawy teoretyczne i Tło technologiczne}

\section{Podstawy teoretyczne} 

\subsection{Pojęcie wirtualnej rzeczywistości}

W celu zdefiniowania pojęcia wirtualnej rzeczywistości w niniejszej pracy konieczne jest odwołanie się do ujęć funkcjonujących w literaturze. Definicje te różnią się zakresem oraz przyjętą perspektywą, od ujęć technicznych po podejścia skoncentrowane na doświadczeniu użytkownika.

W niniejszej pracy przyjęto ujęcie rzeczywistości wirtualnej zaproponowane w książce "\textit{The VR Book}" autorstwa Jasona Jeralda. Rzeczywistość wirtualna rozumiana jest jako w pełni sztuczne, cyfrowe otoczenie generowane komputerowo, które wywołuje u użytkownika poczucie przebywania w innym miejscu lub świecie oraz umożliwia mu bezpośrednie doświadczanie i interakcję w sposób zbliżony do kontaktu z rzeczywistym środowiskiem \cite{Projektowanie-VR}. Przyjęcie tej definicji argumentowane jest tym, że zawiera ona kluczowe pojęcia immersji i obecności, które stanowią podstawowe kryteria oceny jakości doświadczeń w grach VR. W odróżnieniu od tradycyjnych gier komputerowych oraz aplikacji, technologia ta nie ogranicza się tylko do obserwowania obrazu na ekranie, lecz zakłada pełne zaangażowanie użytkownika. 

Dzięki specjalistycznym urządzeniom, takim jak gogle czy kontrolery ruchu, możliwe staje się obserwowanie i oddziaływanie na generowane wirtualnie obiekty w czasie rzeczywistym. Rozwiązania z zakresu VR wykorzystują zaawansowane techniki renderowania grafiki, precyzyjnie śledzą ruch głowy i dłoni, a także uwzględniają dźwięk przestrzenny. Może angażować nie tylko słuch i wzrok, ale również dotyk poprzez kontrolery haptyczne. Wirtualna rzeczywistość oddziałuje na różne \textbf{zmysły}, aby maksymalnie zwiększyć zanurzenie użytkownika i osiągnąć wrażenie, że świat wirtualny jest prawdziwy, a użytkownik naprawdę w nim jest obecny. 

W opisie doświadczeń w grach często stosowane są pojęcia immersja oraz obecność. Słowa te są często mylnie używane jako synonimy, choć odnoszą się do rożnych aspektów VR. Różnice między tymi pojęciami dość trafnie opisał Jerald w swojej książce "\textit{Vr Book}". Zaznacza on, że immersja nawiązuje do cech VR i do tego, jak mocno oddziałuje na zmysły użytkownika. Natomiast obecność odnosi się do subiektywnego wrażenia i odczucia odbiorcy, że znajduje się w świecie wirtualnym. Rozróżnienie to ma ogromne znaczenie dla projektowania interfejsu w grach VR, ponieważ wysoki poziom immersji technicznej nie gwarantuje jeszcze poczucia obecności. W kontekście gier VR pojęcia te nabierają szczególnego znaczenia, ponieważ są bezpośrednio zależne od sposobu i jakości zaprojektowania całego doświadczenia w grze. Przyjęcie tej definicji było istotne z perspektywy dalszej części pracy, ponieważ pojęcia immersji i obecności stanowią podstawę późniejszego badania doświadczeń użytkowników, w szczególności ich subiektywnego poczucia przebywania w wirtualnym środowisku.

\subsection{Ergonomia, percepcja i ograniczenia użytkownika}

W grach komputerowych czy aplikacjach rzadko zdarza się, aby użytkownik odczuwał dyskomfort fizyczny. VR angażuje użytkownika intensywniej niż tradycyjne rozgrywki na komputerze i pozwala mu na pełne zanurzenie się w rozgrywkę. Tworzenie treści do tego środowiska wymaga uwzględnienia ograniczeń percepcyjnych i fizjologicznych człowieka. W literaturze choroba symulatorowa (\textit{VR sickness}) jest opisywana przez autorów jako jedno z możliwych zjawisk towarzyszących korzystaniu z VR. Objawia się najczęściej mdłościami, zawrotami głowy czy ogólna dezorientacją. Według książki "\textit{3D User Interfaces: Theory and Practice}" jest ona spowodowana niespójnością pomiędzy tym, co użytkownik widzi, a tym, co odczuwa jego ciało, czyli brakiem fizycznego czucia ruchu, który odbywa się na oczach odbiorcy \cite{Interfejsy-3D}. Z tego względu autorzy zwracają uwagę na znaczenie takiego czynnika, jak tempo interakcji. Zalecają również ograniczenie gwałtownych zmian perspektywy i widoku oraz przewidywalność zachowań systemu, ponieważ to miedzy innymi one będą miały wpływ na długie i komfortowe korzystanie z urządzenia. 

Zaangażowanie całego ciała w przypadku gier VR mocno angażuje odbiorców, ale nakłada również pewne ograniczenia na projektantów. Jednym z nich jest to, że niewłaściwe ułożenie obiektów na scenie lub wymuszenie, aby dłonie były utrzymane w nienaturalnej pozycji przez dłuższy czas, może prowadzić do szybkiego zmęczenia. Z tego powodu należy uwzględnić naturalny zakres ruchu człowieka, aby wszelkiego rodzaju interakcje mogły być wykonywane bez nadmiernego wysiłku fizycznego \cite{Projektowanie-VR}.

Problematycznym może okazać się samo umieszczenie obiektów interaktywnych w środowisku wirtualnym. Pole widzenia w goglach jest ograniczone nie tylko ze względu na konstrukcję urządzenia, ale również na nasze indywidualne cechy anatomiczne, takie jak rozstaw źrenic. Ma to istotny wpływ na odbiór interfejsu i rozmieszczenie informacji w przestrzeni. Jak wykazują badania autorstwa Sauer i współpracowników o tytule "Assessment of consumer VR
". Pole widzenia, jakie deklarują producenci zestawów VR, różni się od rzeczywistości, a jego zakres zależy od fizjologii odbiorcy. Umieszczenie kluczowych elementów poza zakresem widzialnym będzie prowadzić do ich przeoczenia lub wymuszenia zbędnych ruchów, a w rezultacie do zmęczenia lub irytacji. \cite{Projektowanie-VR,Fov}. Z tego względu należy stosować nie tylko teoretyczne założenia techniczne podane przez producenta, ale również empiryczne ograniczenia odbiorców.

Aby ograniczyć wcześniej wspomniane zmęczenie i zmniejszyć przeciążenie poznawcze podczas projektowania, należy uwzględnić także czas trwania sesji oraz intensywność bodźców. Użytkownik podczas długotrwałego korzystania z VR jest mocno zaangażowany nie tylko sensorycznie, ale również fizycznie. Sprawia to, że nawet po krótkiej sesji może czuć się wyczerpany. Dzieje się to znacznie szybciej niż w przypadku gier na komputer czy konsole. Z tego powodu zaleca się projektowanie doświadczeń w taki sposób, aby umożliwić odpoczynek użytkownikowi. warto również wprowadzić możliwości dostosowania ustawień, takich jak automatyczne trzymanie obiektu czy wysokość kamery. Uwzględnienie tych czynników pozwala ograniczyć negatywne skutki długotrwałego użytkowania VR oraz sprzyja utrzymaniu pozytywnego odbioru doświadczenia przez użytkownika.

Zagadnienia związane z ergonomią, percepcją oraz ograniczeniami użytkownika stanowią istotną bazę wiedzy dla projektowania komfortowych rozwiązań w środowiskach VR na wszystkich etapach. Aspekty te będą następnie uwzględnione w badaniu ankietowym, które pozwoli odnieść opisane w literaturze zjawiska do subiektywnych odczuć użytkowników. Będą miały również bezpośredni wpływ na decyzje projektowe podejmowane podczas tworzenia interakcji w środowisku VR.

\subsection{Typy interfejsów VR (diegetyczne, niediegetyczne, przestrzenne, meta)}

Podział interfejsów użytkownika w VR opiera się na ich relacji względem przestrzeni gry i świata przedstawionego. Zgodnie z uproszczonym modelem projektowym opracowanym przez przez Fagerholta i Lorentzona wyróżnia się 4 główne typy rozwiązań \cite{Fagerholt2009}.

\begin{figure}[!htb]
    \centering
    \includegraphics[width=0.8\textwidth]{images/chapter2/Types.png}
    \caption{Uogólniona klasyfikacja interfejsów użytkownika w grach na podstawie modelu przestrzeni projektowej Fagerholta i Lorentzona.}
    \label{UI-types}
\end{figure}

Pierwszym z nich są interfejsy niediegetyczne, które nie należą do narracji ani do przestrzeni gry i są widoczne tylko dla użytkownika. Najczęściej mogą przybierać formę nieruchomej nakładki (\textit{HUD}), która wyświetla najważniejsze dla odbiorcy informacje, takie jak zdrowie czy czas przed innymi elementami. Pozwalają one na najbardziej bezpośredni dostęp do bieżących informacji i mogą być przystępniejsze dla początkujących, ale ich niewłaściwe umieszczenie może znacząco obniżyć poczucie obecności. 

Drugim typem są interfejsy diegetyczne, czyli fizycznie istniejące w świecie gry elementy, które mają fizyczną obecność. Ich celem jest przekazanie informacji odbiorcy bez konieczności stosowania klasycznych menu. Takie podejście wpływa pozytywnie na realizm i zaangażowanie gracza w rozgrywce. Przykładem takiego rozwiązania może być zegarek lub telefon służący do identyfikacji obiektów w pobliżu.   

Trzecią wyróżnioną kategorią stanowią interfejsy przestrzenne. Mimo że są umieszczone w środowisku gry, nie są częścią świata, a postacie w nim występujące ich nie dostrzegają. Ich główną funkcją jest wizualna pomoc, która ma ułatwić  nawigację i orientację, Mogą obejmować one elementy takie jak linie wskazujące drogę do celu, obrysy przeciwników czy paski zdrowia oraz nazwy unoszące się nad głowami oponentów. Umożliwiają redukcje niepewności gracza, nie zasłaniając jednocześnie pola widzenia. 

Czwartym i ostatnim rodzajem są interfejsy meta. Są one ściśle powiązane z narracją i stanem postaci, nie stanowią jednak fizycznego obiektu w przestrzeni trójwymiarowej. Najczęściej przejawiają się one jako efekty wizualne nałożone na pole widzenia. Mogą przyjmować takie formy jak rozpryski krwi na ekranie w przypadku obniżonego stanu zdrowia czy rozmycie obrazu spowodowanego zmęczeniem. Mają budować więź miedzy graczem a jego wirtualnym awatarem.

Różne typy interfejsów VR implikują odmienne konsekwencje projektowe. Interfejsy silnie zintegrowane ze światem gry mogą sprzyjać poczuciu immersji, lecz jednocześnie będą wymagać od użytkownika większego wysiłku poznawczego oraz nauki nowych schematów interakcji. Z kolei rozwiązania bardziej konwencjonalne, wywodzące się z interfejsów 2D, bywają łatwiejsze do przyswojenia, jednocześnie mogą w większym stopniu zaburzać wrażenie obecności w wirtualnym środowisku \cite{Typy-Ogolne1,Typy-Ogolne2,UI-w-Unity}.

\subsection{Opis typowych sposobów interakcji w VR}

Interakcje to techniki. które pozwalają użytkownikowi na działanie w świecie wirtualnym. Obejmują one wszystkie działania, jakie użytkownik może wykonywać w wirtualnym świecie. jak manipulacje obiektami, przemieszczanie się czy sterowanie interfejsem. Każda metoda interakcji łączy sprzęt. jak na przykład kontrolery z oprogramowaniem, które przekształcają ruchy użytkownika w konkretne akcje w rozgrywce. Jakość tych działań ma bezpośredni wpływ na odbiór całego doświadczenia oraz poziom zaangażowania użytkownika \cite{Interfejsy-3D}.

Jak zauważa Steven Antonio, autor książki "\textit{Enhancing Virtual Reality Experiences with Unity}", tworzenie realistycznych interakcji w VR ma fundamentalne znaczenie dla zapewnienia niezapomnianych, wciągających doświadczeń. Stwierdzenie to jest częściowo trafne, o ile przez realizm rozumiemy spójność bodźców symulacji z mechanizmami percepcji użytkownika. To właśnie ta zgodność determinuje, czy u odbiorcy wystąpią objawy choroby VR.

Interakcje VR można grupować i dzielić na różne sposoby, w zależności od przyjętej perspektywy. W książce "\textit{The VR Book}" autor proponuje podział interakcji na pięć głównych wzorców: selekcji, manipulacji, kontroli punktu widzenia, kontroli pośredniej oraz wzorce złożone \cite{Projektowanie-VR}. Takie podejście koncentruje się głównie na sposobie realizacji interakcji, opisując powtarzalne schematy projektowe wraz z ich zaletami, ograniczeniami oraz przykładami zastosowań. Z kolei autor w książce "\textit{3D User Interfaces: Theory and Practice}" rozważa interakcje z perspektywy zadań wykonywanych przez użytkownika w środowisku wirtualnym, odpowiadając na pytanie, co użytkownik może zrobić w VR \cite{Interfejsy-3D}.

W niniejszej pracy zastosowano podejście zorientowane na użytkownika, dlatego jako główny przyjęto podział zaproponowany przez LaViolę i współautorów, który pozwala przeanalizować interakcję VR w odniesieniu do faktycznych działań użytkownika, ich wpływu na komfort oraz użyteczność doświadczenia. W tym ujęciu wyróżnione zostały cztery główne kategorie interakcji \cite{Interfejsy-3D}.

Autorzy książki "\textit{3D User Interfaces: Theory and Practice}" jako pierwszy typ interakcji wyróżniają selekcję i manipulację. Zadania te są ze sobą ściśle powiązane i w środowisku wirtualnym rzadko występują oddzielnie. Selekcja w kontekście VR polega na wskazywaniu lub identyfikacji obiektu, przy czym może być realizowana za pomocą różnych technik interakcji, takich jak użycie rąk, promienia wskazującego czy kierunku spojrzenia. Jej podstawową funkcją jest określenie, który element środowiska staje się aktualnym celem dalszych działań użytkownika. Z kolei manipulacja obejmuje czynności, jakie użytkownik może wykonać na już wybranym obiekcie. Może to być zmiana jego położenia, orientacji lub wielkości. Selekcja stanowi etap wstępny do manipulacji i jest warunkiem koniecznym do jej rozpoczęcia \cite{Interfejsy-3D}.

Przemieszczanie w środowiskach wirtualnych odnosi się do działań użytkownika związanych z poruszaniem się oraz orientacją w przestrzeni wirtualnej. Składa się ono z dwóch wzajemnie powiązanych komponentów. Pierwszym z nich jest wayfinding, będący aspektem poznawczym nawigacji, odpowiedzialnym za rozumienie przestrzeni, identyfikowanie punktów orientacyjnych oraz planowanie trasy. Drugim jest travel, stanowiący aspekt motoryczny nawigacji, obejmujący fizyczną realizację ruchu oraz sterowanie punktem widzenia. Skuteczna nawigacja wymaga spójnego współdziałania obu tych elementów, ponieważ nawet najbardziej intuicyjne techniki przemieszczania tracą użyteczność bez odpowiedniej orientacji przestrzennej użytkownika \cite{Projektowanie-VR}.

Kontrola systemu w środowiskach trójwymiarowych i wirtualnej rzeczywistości stanowi typ interakcji polegającym na jawnym wydawaniu poleceń systemowi, takich jak uruchamianie funkcji, zmiana trybu interakcji lub modyfikacja stanu systemu. W przeciwieństwie do selekcji, manipulacji czy nawigacji, użytkownik określa jedynie cel działania, pozostawiając sposób jego realizacji po stronie systemu. Nie jest ona główną formą aktywności użytkownika, lecz pełni rolę warstwy sterującej przebiegiem pozostałych interakcji. Projektowanie tych mechanizmów w VR jest szczególnie wymagające, ponieważ konwencjonalne rozwiązania 2D nie przenoszą się bezpośrednio do środowisk immersyjnych, które opierają się na wejściu sześciostopniowym oraz odmiennych uwarunkowaniach percepcyjnych \cite{Projektowanie-VR}.

Z perspektywy projektowej i implementacyjnej interakcje w VR można traktować jako systemy pośredniczące w komunikacji pomiędzy użytkownikiem a aplikacją. Określają one nie tylko zakres możliwych działań, lecz także sposób ich realizacji na poziomie technicznym, obejmujący m.in. obsługę wejścia, reakcje systemu oraz informację zwrotną. Z tego względu sposób zaprojektowania interakcji ma bezpośrednie przełożenie na strukturę i złożoność systemów implementowanych w grze.

Taki sposób klasyfikacji interakcji pozwala na uporządkowanie możliwych działań użytkownika w środowisku VR. Umożliwia on dalsze rozważania nad projektowaniem interfejsów i interakcji w wirtualnej rzeczywistości z perspektywy faktycznych działań wykonywanych przez użytkownika.

\subsection{Klasyczne zasady UX w kontekście środowisk VR}

Chociaż podstawowe zasady projektowania UX (m.in. spójność, hierarchia informacji, czytelność wskazówek interakcyjnych oraz informacja zwrotna) pozostają aktualne także w VR, sposób ich zastosowania wymaga istotnej adaptacji. Niektóre z tych zasad w środowisku VR nabierają większego znaczenia niż w interfejsach dwuwymiarowych, podczas gdy inne wymagają odmiennego sposobu zastosowania. W środowisku wirtualnym użytkownik jest odcięty od bodźców świata rzeczywistego i nie może wspierać się naturalnymi wskazówkami otoczenia, dlatego kluczowe staje się projektowanie pod kątem "odkrywalności" otoczenia. Produkt powinien dawać możliwość samodzielnego badania, co dany element robi, jak działa oraz co można z nim zrobić. 

Z perspektywy Dona Normana głównym winowajcą nie jest użytkownik, tylko produkt, który nie komunikuje w sposób zrozumiały wszystkich swoich możliwości. Jeśli potrzebna jest bezpośrednia instrukcja użytkowania, problemem jest niewystarczająca ilość czytelnych znaczników. W VR jest to niezwykle istotne, ponieważ odbiorca nie może spojrzeć poza interfejs i skorzystać z podpowiedzi obecnych w świecie rzeczywistym. Elementy z którymi możliwe jest wchodzenie w interakcję, powinny być dobrze widoczne poprzez zastosowanie afordancji i czytelnych znaczników, które będą sygnalizować funkcje danego przedmiotu już przed wejściem z nim w interakcje \cite{Odczucie-gry,Projektowanie-VR}.

Określenie \textit{afordancje} opisuje relację między właściwościami systemu lub przedmiotu a możliwościami działania, jakie oferują użytkownikowi. Projektanci powinni w ten sposób kształtować interakcje z obiektami w VR, aby wszystkie pożądane działania, które użytkownik mógłby chcieć wykonać, były łatwe do osiągnięcia. Niechciane akcje powinny być niemożliwe lub trudne do wykonania, a tego właśnie służą "\textit{ograniczenia}". Nie tylko limitują one to, co użytkownik może zrobić, ale mogą również posłużyć do zmniejszenia liczby błędów i uproszczenia sposobu, w jaki interfejs zostanie odebrany \cite{Odczucie-gry,Projektowanie-VR}.

Zarówno Jerald, jak i Norman uwzględniają w swoich książkach, że istotne jest "\textit{Mapowanie}". Jest to związek między działaniem użytkownika a jego skutkiem. Naturalne mapowanie można poniekąd porównać do realizmu, gdyż kierunek ruchu w środowisku wirtualnym powinien odpowiadać kierunkowi ruchu kontrolera. Zastosowanie się do tej zasady znacząco obniża koszt poznawczy i może pozytywnie wpłynąć na prędkość nauki obsługi interfejsu. Don Norman w swoim tytule podaje wiele przykładów, że nawet działający mechanizm może być kłopotliwy w korzystaniu, jeśli nie zgadza się z modelem mentalnym użytkownika. Brak naturalnego mapowania w VR może bardzo szybko doprowadzić do dezorientacji, poczucia utraty kontroli, a nawet do objawów choroby symulatorowej \cite{Odczucie-gry,Projektowanie-VR}.

Kolejnym równie ważnym aspektem jest konwencja i standaryzacja, czyli najczęściej stosowany schemat działania. Oznacza to, że użytkownicy, korzystając z gier lub symulatorów, mają już swoje przewidywania, jak dany system będzie działać i jak wykonywać w nim akcje. Każde odchylenie od ustalonych norm będzie szybko zauważone i może zwiększać obciążenie poznawcze lub wywołać frustracje. Z tego powodu zmiany należy wprowadzać ostrożnie i stopniowo \cite{Odbior-gry}.

Ostatnim zagadnieniem jest informacja Zwrotna ("\textit{feedback}"). Pełni ona jednocześnie dwie funkcje. Informuje o skutkach wykonanej właśnie czynności oraz daje poczucie panowania nad sytuacją. W VR musi ona być szybka i jednoznaczna, ponieważ jakiekolwiek opóźnienia mogą być odebrane jako błąd systemu. Nawet brak informacji zwrotnej może zostać zinterpretowany przez odbiorce jako awaria i doprowadzić do frustracji oraz zaburzać naturalny przepływ doświadczenia.

Podsumowując, adaptacja klasycznych zasad UX do VR nie polega na prostym przeniesieniu rozwiązań z interfejsów 2D do przestrzeni trójwymiarowej, lecz na ich świadomym dopasowaniu do projektowanego środowiska i zaplanowaniu relacji między użytkownikiem a systemem. Czytelne "\textit{affordance}", jednoznaczne znaczniki, naturalne mapowania, konsekwentne ograniczenia oraz natychmiastowa informacja zwrotna tworzą spójny model interakcji, który wspiera zarówno efektywność zadaniową, jak i pozytywny odbiór emocjonalny doświadczenia.

\section{Tło technologiczne VR} 

Kluczowym etapem projektowania tej pracy był wybór odpowiednich narzędzi umożliwiających prototypowanie, implementację oraz testowanie interfejsu w środowisku VR. Wybór używanej technologii jest kluczowy, aby zapewnić wysoką jakość doświadczeń użytkownika oraz zachować efektywność samego procesu projektowania.

\subsection{Silniki gier i środowiska VR}

\subsubsection{Unity 3D}
\textit{Unity} to jeden z najpopularniejszych wieloplatformowych silników do tworzenia gier wideo oraz interaktywnych aplikacji (rysunek \ref{unity_engine_example}). Umożliwia zaawansowaną obsługę grafiki i fizyki, a w samym edytorze udostępniono rozbudowany zestaw narzędzi do projektowania scen, animacji i efektów graficznych. Silnik zawiera bardzo rozbudowaną bazę bibliotek i dodatkowych paczek z zasobami, co pozwala w szybki sposób dodać integrację na każdą platformę i urządzenie. Proces programowania odbywa się przy użyciu języka \textit{C\#}, co również jest dużą zaletą, gdyż poziom wejścia jest niższy niż w przypadku języków niższego poziomu, takich jak \textit{C++}.

W silniku Unity dodano również wsparcie dla wirtualnej oraz rozszerzonej rzeczywistości, co przekłada się na szerokie zastosowanie tej technologii w szkoleniach, symulacjach czy działaniach marketingowych. Oprócz możliwości tworzenia zaawansowanych pod względem wizualnym scen, istotna jest też opcja łatwej integracji gotowych wtyczek obsługujących urządzenia AR i VR od różnych producentów. Rozbudowany system oświetlenia oraz post-processingu zapewnia szeroki wachlarz opcji i umożliwia dostosowanie grafiki do dedykowanych urządzeń, dzięki czemu można uzyskać zadowalającą jakość oraz odpowiednią optymalizację na urządzeniach mobilnych i wysoce realistyczną grafikę na komputerach osobistych.

\begin{figure}[!htb]
    \centering
    \includegraphics[width=0.8\textwidth]{images/unity.png}
    \caption{Przykładowy wygląd edytora Unity 6}
    \label{unity_engine_example}
\end{figure}

Regularne aktualizacje silnika rozszerzają jego funkcjonalność i dodają nowe rozwiązania, takie jak obsługa technologii \textit{ray tracingu} czy \textit{DLSS}. Stały rozwój sprawia, że Unity zachowuje elastyczność w obliczu zmieniających się potrzeb rynku, a jego wszechstronność umożliwia realizację nawet najbardziej rozbudowanych koncepcji interaktywnych.

\subsubsection{Unreal Engine}
\textit{Unreal Engine} to wieloplatformowy silnik gier stworzony przez \textit{Epic Games}, ceniony przede wszystkim za zaawansowane możliwości graficzne i fotorealistyczny rendering. Środowisko edytora zapewnia bogaty zestaw narzędzi do edycji scen, tworzenia materiałów, animacji oraz efektów post-processingu. Wyróżniającą cechą silnika jest jego nowoczesny sposób renderowania, obejmujący między innymi technologię dynamicznego globalnego oświetlenia i odbić \textit{Lumen} oraz system wirtualizowanej geometrii \textit{Nanite}. Silnik obsługuje ponadto sprzętowy \textit{ray tracing}, umożliwiający generowanie bardziej wiarygodnych cieni i odbić na zgodnym sprzęcie (rysunek \ref{unreal_engine_example}).

\begin{figure}[!htb]
    \centering
    \includegraphics[width=0.8\textwidth]{images/chapter2/unreal.jpeg}
    \caption{Przykładowy wygląd edytora Unreal Engine 5}
    \label{unreal_engine_example}
\end{figure}

Logikę gry można implementować zarówno w języku \textit{C++}, jak i za pomocą wizualnego systemu skryptowego \textit{Blueprints}. Takie podejście ułatwia współpracę między programistami a projektantami mechanik. W kontekście zastosowań immersyjnych istotna jest obecność wtyczki \textit{OpenXR}, która zapewnia zunifikowany interfejs komunikacji z urządzeniami i platformami zgodnymi ze standardem OpenXR.\cite{UE_XR_OpenXR}

\subsubsection{Godot Engine}
\textit{Godot Engine} to bezpłatny silnik o otwartym kodzie źródłowym, przeznaczony do tworzenia gier w 2D i 3D. Jest rozwijany społecznościowo i dostępny na licznych platformach. Oferuje kompletny edytor oraz elastyczny system wtyczek, co pozwala prowadzić projekt bez konieczności posiadania komercyjnych licencji ani uzależnienia od zamkniętych ekosystemów. Podstawowym językiem skryptowym jest \textit{GDScript}, zaprojektowany z myślą o szybkim prototypowaniu wewnątrz silnika. W określonych konfiguracjach możliwe jest również wykorzystanie \textit{C\#} (rysunek \ref{godot_engine_example}).

\begin{figure}[!htb]
    \centering
    \includegraphics[width=0.8\textwidth]{images/chapter2/godot.jpg}
    \caption{Przykładowy wygląd edytora Godot 4}
    \label{godot_engine_example}
\end{figure}

Pod względem grafiki i wydajności Godot udostępnia kilka trybów renderowania oraz obsługę współczesnych interfejsów graficznych, takich jak \textit{Vulkan} oraz sterowniki zależne od platformy docelowej. Pozwala to dostosować rendering do możliwości danego sprzętu. W obszarze technologii \textit{XR}, począwszy od wersji 4.0, silnik posiada natywną obsługę \textit{OpenXR}. Funkcje specyficzne dla poszczególnych producentów sprzętu mogą być realizowane przez osobne wtyczki rozszerzające ten standard.\cite{GodotOpenXRVendors}

\subsection{Standardy i warstwy uruchomieniowe VR}

W praktyce każdy producent zestawów VR powinien dostarczyć deweloperom narzędzia umożliwiające tworzenie i uruchamianie aplikacji na swoim sprzęcie (np. biblioteki, środowiska uruchomieniowe i dokumentację). Brak takiej warstwy integracyjnej wymuszałby utrzymywanie odrębnych ścieżek implementacyjnych dla kolejnych urządzeń, co znacząco zwiększa koszty rozwoju i utrzymania oprogramowania.

We wczesnym okresie popularyzacji konsumenckiego VR rynek był silnie rozproszony: poszczególne platformy dostarczały własne zestawy interfejsów programistycznych. Utrudniało to wytworzenie jednej wersji aplikacji działającej spójnie na różnych zestawach. Rozwiązaniem tej fragmentacji na komputerach osobistych było m.in. \textit{OpenVR} udostępnione przez firmę \textit{Valve} jako warstwa \textit{API} oraz runtime związany z ekosystemem SteamVR, projektowany do obsługi sprzętu wielu producentów.\cite{OpenVRRelease2015}

\textit{OpenVR} umożliwiało w praktyce uruchamianie aplikacji na różnych zestawach VR dostępnych na rynku PC, w tym na \textit{HTC Vive} oraz \textit{Oculus Rift}. Przejęcie Oculus VR przez Facebook (obecnie \textit{Meta}) w 2014 roku dodatkowo podkreśliło, że istotna część rynku VR rozwija się w ramach konkurujących ekosystemów platformowych.\cite{FacebookOculus2014}

W kolejnych latach podjęto próbę ustandaryzowania interfejsu na poziomie całej branży rozszerzonej rzeczywistości (\textit{XR}), obejmującej \textit{VR}, \textit{AR} oraz \textit{MR}. Organizacja non-profit \textit{Khronos Group}, odpowiedzialna za rozwój otwartych, wolnych od opłat licencyjnych standardów API, opracowała standard \textit{OpenXR}.\cite{KhronosAbout,OpenXRAbout} Definiuje on wspólny interfejs dla aplikacji \textit{XR}. Celem było umożliwienie aplikacjom komunikacji z ujednoliconym API, przy jednoczesnym przeniesieniu różnic sprzętowych i platformowych do warstwy \textit{runtime} dostarczanej przez producenta lub ekosystem.\cite{OpenXR10Press}

Z perspektywy silnika \textit{Unity} przejście to było widoczne również w sposobie dostarczania wsparcia XR. Starsze rozwiązania (np. dedykowane pakiety pod konkretne API, takie jak OpenVR) były stopniowo wypierane przez architekturę \textit{XR Plug-in Management}, w ramach której aktywuje się dostawców XR jako moduły. Jednym z głównych kierunków stała się integracja oparta o \textit{OpenXR}.\cite{UnityOpenVRDeprecated,UnityXRManagement,UnityOpenXRPlugin}

\subsection{Charakterystyka platformy sprzętowej}

W przypadku aplikacji VR termin \textit{„platforma sprzętowa”} odnosi się nie tylko do gogli, ale również do kontrolerów, systemów śledzenia ruchu oraz sposobu łączności z komputerem bądź urządzeniem mobilnym. Obejmuje także warstwę uruchomieniową (\textit{runtime}), która dostarcza aplikacji informacji o pozycji i orientacji użytkownika w przestrzeni. Centralnym aspektem jest mechanizm śledzenia (\textit{tracking}), który może opierać się na zewnętrznych stacjach bazowych albo na alternatywie wykorzystującej kamery i czujniki zamontowane bezpośrednio na urządzeniu. Dobór metody śledzenia wpływa na dokładność odwzorowania ruchu, stabilność wyświetlanego obrazu i wygodę użytkowania. Wyznacza też wymogi dotyczące aranżacji przestrzeni, w której przeprowadzane są testy.

Ważnym składnikiem platformy są urządzenia wejściowe, zwłaszcza kontrolery ruchowe oraz mechanizmy haptycznego sprzężenia zwrotnego, stanowiące w VR główny kanał interakcji z aplikacją. Różnice w budowie kontrolerów (ergonomia, liczba przycisków, obecność drążków analogowych czy jakość wibracji) mogą istotnie oddziaływać na sposób projektowania interfejsu oraz manipulowania obiektami w scenie. Niektóre zestawy oferują ponadto śledzenie dłoni (\textit{hand tracking}) lub śledzenie wzroku (\textit{eye tracking}). Otwiera to drogę do bardziej intuicyjnych form interakcji i dodatkowych metod selekcji, ale wiąże się z większą złożonością implementacyjną i wyższymi wymaganiami sprzętowymi.

Z perspektywy wierności percepcji kluczowe znaczenie mają parametry wyświetlacza, w tym rozdzielczość, częstotliwość odświeżania oraz pole widzenia (\textit{FOV}). Parametry te bezpośrednio oddziałują na czytelność elementów interfejsu, postrzeganie szczegółów sceny oraz skłonność do wystąpienia dyskomfortu, takiego jak zmęczenie oczu czy objawy choroby lokomocyjnej. W związku z tym projektowanie interfejsu VR powinno brać pod uwagę ograniczenia wynikające z charakterystyki docelowego wyświetlacza oraz przewidywanej odległości, z jakiej użytkownik obserwuje elementy UI.