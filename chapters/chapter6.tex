\chapter{Badania końcowe i wnioski projektowe}

\section{Cel eksperymentu}
 Celem przeprowadzenia badania było empiryczne porównanie wpływu dwóch różnych podejść do projektowania interfejsu użytkownika oraz wybranych mechanizmów interakcji w wirtualnym środowisku. Analiza obejmowała typ interfejsu, sposób przemieszczania się oraz intensywność informacji zwrotnej, czyli liczbę kanałów komunikujących reakcje systemu na wykonane zadania. Miały również ocenić poziom immersji i poczucia obecności użytkownika podczas wykonywania czterech zadań. Porównanie obu wariantów przeprowadzono w kontrolowanych warunkach, zachowując jednocześnie identyczną strukturę zadań, układu i celów do wykonania. Rożnicę pomiędzy wariantem 1 i 2 ograniczono do badanych zmiennych, aby następnie zestawić zachowania użytkowników i ocenić wpływ decyzji projektowych na przebieg rozgrywki 
 

\section{Metodologia badań}

 Zakres badanych obszarów został wyselekcjonowany na podstawie wcześniejszych wyników analizy heurystycznej oraz badania ankietowego. Do badań włączono te aspekty interakcji, które okazały się problematyczne lub niejednoznaczne w obu etapach analizy. Jednym z najczęściej wskazywanych problemów w doświadczeniach VR były mechanizmy lokomocji. Informacja zwrotna, mimo że nie była wskazywana przez użytkowników jako obszar krytyczny, została uwzględniona w badaniu ze względu na wyniki analizy heurystycznej. W większości analizowanych gier VR była ona ograniczona lub pomijana, dlatego zdecydowano się na jej uwzględnienie w badaniu końcowym w celu oceny wpływu intensywności informacji zwrotnej na jakość rozgrywki i przebieg interakcji.
 
 Z zakresu badań świadomie zrezygnowano również z mechaniki ekwipunku, pomimo problemów wykrytych w pierwszym etapie analizy. Decyzja ta wynikała z faktu, że zidentyfikowane trudności dotyczyły wyłącznie jednego typu rozwiązania, tj. ekwipunku realizowanego w formie panelu osadzonego w przestrzeni. Pozostałe warianty uzyskały dobre oceny użyteczności i były wskazywane w badaniu ankietowym jako rozwiązania udane. Uznano zatem, że uwzględnienie mechaniki ekwipunku nie wniesie istotnej wartości do dalszych badań, a zakres pracy zawężono do obszarów generujących największe i najczęściej występujące problemy użytkowe. Dodatkowo obszar ten pokrywał się częściowo funkcjonalnie z interfejsem panelowym, który został już objęty zakresem badania.
W celu wyeliminowania efektu uczenia się badanie zostało zaprojektowane zgodnie z modelem międzygrupowym i przeprowadzone stacjonarnie. Uczestnicy zostali podzieleni na dwie niezależne grupy badawcze, a każda z nich testowała wyłącznie jeden wariant interfejsu. Wybór projektu międzygrupowego został podyktowany obawą, że schemat wewnątrzgrupowy mógłby prowadzić do zapamiętania schematu zadania przez użytkowników i zakłócenia wyników. W badaniu wzięło udział 10 dorosłych osób a rekrutacja odbywała się wśród znajomych i rodziny. Wybrano 6 osób, które miały niewielką styczność z projektowanym środowiskiem, nieprzekraczającą około 30 minut wcześniejszego kontaktu, oraz 4 posiadające minimum 12h doświadczenia z urządzeniami VR. Przydzielenie osob doświadczonych i początkujących odbyło się losowo tak aby powstałe grupy były równo zróżnicowane pod względem poziomu zaawansowania z technologią VR. Decyzja o takim podziale wynikała z wyników ankiet w których to duża część ankietowanych wskazała na jednokrotne korzystanie z tej technologi. Każda z nich zawierała 3 użytkowników początkujących i 2 zaawansowanych. Zapewniło to kontrole nad możliwym rożnym poziomem doświadczenia osób badanych i pozwoliło na sprawdzenie czy wcześniejsze doświadczenie wpłynie na odbiór danego wariantu. 

	\begin{figure}[!htb]
    \centering
    \includegraphics[width=1
    \textwidth]{images/chapter6/1.png}
    \caption{Schemat przebiegu eksperymentu międzygrupowego w VR., (2026); źródło: własne}
    \end{figure}


Zaprojektowane zadania nie miały określonego limitu czasu na zadanie aby umożliwić naturalne zachowanie i swobodną eksploracje. W przypadku trudności badanie nie było przerywane, ale dopuszczono przerwy spowodowane złym samopoczuciem. Aby nie wpływać na przebieg badania podczas wykonywania zadań nie udzielano żadnych podpowiedzi. Dane ilościowe były zapisywane automatycznie przez system. Badanie opierało się wyłącznie na metrykach zapisywanych przez system i notatkach obserwacyjnych.

Zastosowanie tej metody miało na celu weryfikację, w jakim stopniu dany prototyp wspiera użytkownika w wykonywaniu założonych czynności \cite{Interakcja}. Dodatkową zaletą obserwacji była możliwość identyfikacji przyczyn zróżnicowanych wyników między poszczególnymi scenariuszami, co pozwoliło określić, które elementy środowiska sprzyjają efektywnej interakcji, a które wymagają dalszych modyfikacji i optymalizacji.

\section{Wyniki badań}
Czas wykonania zadań rejestrowano automatycznie przez system z dokładnością do sekundy. Jest to podstawowa metryka użyteczności, opisująca czas potrzebny użytkownikowi do ukończenia zdefiniowanego zadania \cite{Pomiary}.
\subsection{Wyniki Badań}



\section{Interpretacja wyników}
Czas liczony był w sekundach z dokładnoscią do ..  książce xyz 
rodział 8 3
rekomendowano użcyie jednej z 3 metryk 
\cite{Pomiary}

\section{Ograniczenia badań}

Przeprowadzone badanie wiąże sie z kilkoma ograniczeniami które wpływają na zakres otrzymanych wniosków. Ograniczenia przeprowadzonego badania wynikają przede wszystkim z dostępnych zasobów. Były one świadomymi decyzjami wpływającymi na zakres otrzymanych wniosków. Próba badawcza wynosiła 10 osób co pozwalało na wychwycenie tendencji i różnic pomiędzy poszczególnymi wariantami. Nie daje jednak podstaw do wyciągania wniosków dotyczących calej poopulacji. Zastosowanie schematu międzygrupowego wyeliminowała efekt uczenia się, ale zwiększyła wpływ indywidualnych różnic miedzy użytkownikami. Pomimo zachowanie równowagi w każdej z grup, mała liczba badanych ograniczyła możliwość przeprowadzenie dogłębnej analizy wpływu poziomu zaawansowania. Badania zostały przeprowadzone w warunkach domowych na jednym sprzęcie ---- Dostępne były inne zestawy VR jednak nie oferowały one funkcji ktore mogłyby znacząco wpłynąć na zakres pomiarów fizjologicznych. Przeprowadznie testów w warunkach domowych pozwoliło na większą elastyczność organizacyjną i umożliwiło dostosowanie terminów sesji do uczestników. Uzyskane wyniki odnoszą się do konkretnej konfiguracji testowej

- brak gogli z możliwościa śledzenia gałek ocznych
- brak aparatury do mierzenia tętna

\section{Wnioski projektowe i podsumowanie pracy}



