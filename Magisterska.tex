\documentclass[xodstep]{wnspt}

\author   {XYZ}
\nralbumu {77071} 
\kierunek {Informatyka}
\specjalnosc {Tworzenie gier komputerowych}

\date     {2024}
\miejsce {Cz\k{e}stochowa}
%\instytut {Zak\l{}adzie Informatyki Stosowanej}
\opiekun  {dr hab. Andrzej Zbrzezny}

\usepackage{amsmath}
\usepackage{amsfonts}
\usepackage{amsthm}
\usepackage{amssymb}
\usepackage[T1]{fontenc}
\usepackage[utf8]{inputenc}
\usepackage[polish]{babel}
\usepackage{polski}
\usepackage{colortbl}
\usepackage{hyperref}
\usepackage{url}
\usepackage{setspace}
\usepackage{indentfirst}
\usepackage{listingsutf8}
\usepackage{beramono}
\usepackage[section]{placeins}
\usepackage{csquotes}
\usepackage{carlito} % Dodanie czcionki Carlito
\usepackage[%backend=biber,refsegment=section,defernumbers=true,]{biblatex}

\lstset{ %
language=c++,                  % choose the language of the code
basicstyle=\ttfamily,           % the fonts that are used for the code
numbers=left,                   % where to put the line-numbers
numberstyle=\footnotesize,      % the size of the fonts that are used for the line-numbers
stepnumber=1,                   % the step between two line-numbers. If it's 1 each line will be
numbersep=5pt,                  % how far the line-numbers are from the code
showspaces=false,               % show spaces adding particular underscores
showstringspaces=false,         % underline spaces within strings
showtabs=false,                 % show tabs within strings adding particular underscores
frame=single,                   % adds a frame around the code
tabsize=4,                      % sets default tabsize to 2 spaces
captionpos=b,                   % sets the caption-position to bottom
breaklines=true,                % sets automatic line breaking
breakatwhitespace=false,        % sets if automatic breaks should only happen at whitespace
escapeinside={\%*}{*)}          % if you want to add a comment within your code
}

\newcommand{\R}{mathbb{R}}

\renewcommand{\lstlistlistingname}{Spis listingów}
\renewcommand{\lstlistingname}{Listing}

\newtheorem{lemat}{Lemat}
\newtheorem{twierdzenie}{Twierdzenie}

\title{\begin{LARGE}Optymalizacja interfejsów użytkownika w grach wirtualnej rzeczywistości: Perspektywa programisty i projektanta UX/UI
\\{~}
Optimizing user interfaces in virtual reality games: A developer and UX/UI designer's perspective
\end{LARGE}}

\frenchspacing
\begin{document}
\begin{abstract}
Celem pracy dyplomowej było opisanie i opracowanie projektu interfejsu użytkownika
Dokonano analizy konkurencji, zaprojektowano 
Przebudowano gotowe systemy
Wdrożony system został przetestowany potwierdzając poprawne działanie.


Opisano również dalsze możliwości rozwoju gry.
\end{abstract}

\keywords{Izometria,Język Skryptowy, interfejs graficzny, aplikacja internetowa}
\maketitle

\onehalfspacing

\introduction

W dzisiejszych czasach gry komputerowe stały się powszechną formą spędzania wolnego czasu. Nic więc dziwnego, że firmy prześcigają się w rozwijaniu technologii i zapewnianiu konsumentom coraz to lepszych wrażeń podczas grania. Szczególnie dla odbiorców, oprócz szczegółowej grafiki oraz złożonych mechanik, ważne są także kosmetyczne elementy, które pozwalają bardziej zagłębić się w przedstawiony świat.


	
	Układ tej pracy jest następujący. W rozdziale pierwszym opisane zostały narzędzia i programy wykorzystane podczas tworzenia pracy. Pokazano również grę, która stała się inspiracją do stworzenia systemu zmiany pory roku oraz przeprowadzono analizę konkurencyjnych gier istniejących już na rynku. W rozdziale drugim opisane zostały często używane w grach style graficzne. Następnie przeanalizowano je pod kątem ich rozmiaru oraz innych parametrów, by następnie dokonać wyboru stylistyki pasującej do rozgrywki pod względem graficznym i implementacyjnym. 
W rozdziale trzecim skupiono się na

\chapter{Wprowadzenie do tematyki pracy, Analiza rozwiązań dostępnych na rynku, wykorzystane narzędzia}

\section {Przegląd aktualnych technologii VR}
\section {Analiza istniejących aplikacji VR w kontekście interakcji użytkownika}
\section {Identyfikacja najlepszych praktyk i innowacji w projektowaniu interfejsow VR}

Pierwszym etapem projektowania tej pracy był wybór odpowiednich narzędzi umożliwiających prototypownie, implementację oraz testowanie interfejsu w środowisku VR. Wybór używanej technologii jest kluczowy, aby zapewnić wysoką jakość doświadczeń użytkownika oraz zachować efektywność samego procesu projektowania.

\subsection{Figma}
Figma to podstawowe narzędzie służące do projektowania interfejsów użytkownika, jednocześnie umożliwia tworzenie interaktywnych klikalnych prototypów aplikacji. Jest to aktualnie jedno z najbardziej popularnych i bezpłatnych narzędzi w branży UX/UI. W projekcie Figma zostanie wykorzystana do stworzenia wstępnych projektów i prototypów interfejsów, a następnie do przetestowania rożnych układów elementów interfejsu. 
%(tu będzie dokumentacja figmy) 
\subsection{Unity}
\subsection{Sprzęt VR}
\subsection{Google Forms i inne narzędzia do analizy UX/UI}








%Tworzenie interfejsów użytkownika w środowisku wirtualnej rzeczywistości wiąże się z wieloma trudnościami, które nie występują w tradycyjnych aplikacjach. Ze względu na trójwymiarowy wymiar wirtualnej rzeczywistości konieczne będzie zastosowanie nowych zasad ergonomii, ktore umożliwią użytkownikom komfortową, intuicyjną oraz płynną interakcję z produktem. Przy projektowaniu należy wziąć pod uwagę takie aspekty jak ograniczenia technologiczne i sensoryczne jak np. pole widzenia czy ograniczenia motoryczne użytkowników. 

\section{Ogólne zasady projektowania interfejsów użytkownika}
\chapter{Podstawy teoretyczne i Tło technologiczne}

\section{Podstawy teoretyczne} 


\subsection{Definicje, pojęcia i klasyfikacje VR}
Aby uzasadnić potrzebe projektowania interfejsów i interakcji w grach VR w odmienny sposób niż w przypadku gier 2D czy 3D, należy najpierw zrozumieć czym jest tak naprawde VR. 
W niniejszej pracy przyjęto ujęcie rzeczywistości wirtualnej zaproponowane w książce The VR Book autorstwa Jasona Jeralda. Rzeczywistość wirtualna rozumiana jest jako w pełni sztuczne, cyfrowe otoczenie generowane komputerowo, które wywołuje u użytkownika poczucie przebywania w innym miejscu lub świecie oraz umożliwia mu bezpośrednie doświadczanie i interakcję w sposób zbliżony do kontaktu z rzeczywistym środowiskiem (Jerald, The VR Book). W odróznieniu od tradycyjnych gier komputerowych oraz aplikacji, technologia ta nie ogranicza się tylko do obserwowania obrazu na ekranie lecz zakłada pełne zaangażowanie użytkownika. 

Dzięki specjalistycznym urządzeniom, takim jak gogle czy kontrolery ruchu, możliwe staje się obserwowanie i oddziaływanie na generowane wirtualnie obiekty w czasie rzeczywistym. Rozwiązania z zakresu VR wykorzystują zaawansowane techniki renderowania grafiki, precyzyjnie śledzą ruch głowy i dłoni, a także uwzględniają dźwięk przestrzenny. Może angażować nie tylko słuch i wzrok ale również dotyk poprzez kontrolery haptyczne. Wirtualna rzeczywistość odziałowywuje na różne \textbf{zmysły}, aby maksymalnie zwiększyć zanurzenie użytkownika i osiągnięcie wrażenia, że świat wirtualny jest prawdziwy a użytkownik jest w nim naprawdę obecny. 

Do opisu doświadczeń w grach często stosowane jest pojęcie immersja oraz obecność. Słowa te są czesto mylnie stosowane jako synonimy choć odnoszą się do rożnych aspektów VR. Różnice między tymi pojęciami dość trafnie opisał Jerald w swojej książce book. Zaznacza on, że Imersja nawiązuje do cech VR, do tego w jak mocno oddziałowują na na zmysły użytkownika. Natomiast obecność odnosi się do subiektywnego wrażenia i odczucia użytkownika że znajduje się w świecie wirtualnym. W kontekscie gier VR te słowa mają szczególne znaczenie, są zależne od tego jak zaprojektowana jest cała gra. 

\begin{figure}[!htb]
    \centering
    \includegraphics[width=0.8\textwidth]{images/vr_example.jpg}
    \caption{Symulator medyczny służący do szkolenia lekarzy}
    \label{vr_example}
\end{figure}\textbf{}


\subsection{Ergonomia, percepcja i ograniczenia użytkownika}

W grach komputerowych czy aplikacjach rzadko zdarza się aby użytkownik odczuwał dyskomfort fizyczny. VR angażuje urzytkownika intensywniej niż tradycyjne rozgrywki na komputer i pozwala użytkownikowi na pełne zanurzenie się w rozgrywkę. Tworzenie treści do tego środowiska wymaga uwzględnienia ograniczeń percepcyjnych i fizjologicznych człowieka. Jednym z efektów braku takiego podejścia jest wystąpienie objawów choroby symulatorowej (VR sickness). Objawia się najczęściej mdłościami, zawrotami głowy czy ogólna dezorientacją. Według ksiązki 3D User Interfaces: Theory and Practice (book) (3D user Interfaces Theory and Practice) jest ona spowodowana niespójnością pomiędzy tym, co użytkownik widzi czyli ruch w świecie wirtualnym, a tym, co odczuwa jego ciało czyli brak fizycznego ruchu. Z tego wzgledu bardzo istotne jest tempo interakcji, ograniczenie gwałtwonych zmian perspektywy i widoku oraz przewidywalność zachowań systemu ponieważ to miedzy innymi one będą miały wpływ na długie i komfortowe korzystanie z urządzenia. 

\begin{figure}[!htb]
    \centering
    \includegraphics[width=0.8\textwidth]{images/VRSICK.png}
    \caption{VR sickness}
    \label{unity_engine_example}
\end{figure}
(3D user Interfaces Theory and Practice)


Zaangażowanie całego ciała w przypadku gier VR mocno angażuje odbiorców ale nakłada również pewne ograniczenia na projektantów. Jednym z nich jest to że niewłasciwe ułożenie obiektów na scenie lub wymuszenie utrzymywania dłoni w nienaturalnej pozycji przez dłuższy czas może prowadzić do szybkiego zmęczenia. Z tego powody należy uwzględnić naturalny zakres ruchu człowieka aby wszelkiego rodzaju interakcje mogły być wykonywane bez nadmiernego wysiłku fizycznego.  (book) jerald

Problematycznym może okazać się samo umieszczenie obiektów interaktywnych w środowisko wirtualnym. Pole widzenia w googlach jest ograniczone nie tylko ze względu na konstrukcje urządzeni ale również naszych indywidualnych cech anatomicznych jak rozstaw źrenic. Ma to istotny wpływ na odbiór interfejsu i rozmieszczenie informacji w przestrzeni. Jak wykazują badania  Sauer i współpracowników (2022),  article (Assessment of consumer VR). pole widzenia jakie deklarują producenci zestawów VR jest inne niż rzeczywistosci a jego zakres zależy fizjologi odbiorcy. Umieszczenie kluczowych elementów poza zakresem widzialnym będzie prowadził do ich przeoczenia lub wymuszenia zbędnych ruchów i w rezultacie zmęczenia lub irytacji. 
%Ten fragment cały ma informacje z dwoch ksiazek i jednego artykulu i nie wiem jak cytować  (Jerald, 2015; LaViola et al., 2017; Sauer et al., 2022).
Z tego względu należy stosować nie tylko teoretyczne założenia techniczne podane przez producenta ale również empiryczne ograniczenia odbiorców.

Aby ograniczyć wcześniej wspomniane zmęczenie i zmniejszyć przeciążenie poznawcze podczas projektowania należy uwzglednić również czas trwania sesji i intensywność bodzców. Użytkownik podczas długotrwałego korzystania z vr jest mocno zaangażowany nie tylko sensorycznie ale również fizycznie. Sprawia to że nawet po krótkiej sesji może czuć sie wyczerpany. Dzieje się to znacznie szybciej niż w przypadku gier na komputer czy konsole. Z tego powodu zaleca się projektowanie doświadczeń w taki sposób by umożliwić odpoczynek użytkownika. warto również wprowadzenić możliwości dostosowania ustawień takich jak automatyczne trzymanie obiektu czy wysokość kamery. Uwzględnienie tych czynników pozwala ograniczyć negatywne skutki długotrwałego użytkowania VR oraz sprzyja utrzymaniu pozytywnego odbioru doświadczenia przez użytkownika.

 

\subsection{Typy interfejsów VR (2D, 3D, diegetyczne, systemowe)}

Projektowanie interfejsu do VR może odbywać się na różne sposoby, różniące się stopniem integracji z wirtualnym środowiskiem i tym jak prezentowane są istotne informacje. Aby odpowiednio przygotować się do projektowania, konieczna będzie analiza rozwiązań stosowanych w istniejących już grach i aplikacjach. Twórcy gier podejmują różne decyzje dotyczące formy interfejsu użytkownika, wynikające z odmiennych priorytetów projektowych, od dążenia do maksymalnej immersji poprzez silną integrację interfejsu ze światem gry, po wykorzystanie bardziej tradycyjnych rozwiązań w postaci paneli menu i lewitujących w przestrzeni wirtualnej elementów interfejsu. W pierwszej kolejności warto poznać podstawowe rodzaje interfejsów oraz ich charakterystykę. Pozwala to lepiej zrozumieć, w jakich sytuacjach i w jakim celu poszczególne rozwiązania są wykorzystywane. W niniejszej analizie przyjęto podział interfejsów na cztery główne typy:

Interfejs jako część świata gry, Klasyczny panel menu (Non-diegetic UI), Interfejs przestrzenny oraz Meta.
Pierwszy rodzaj polega na wpleceniu interfejsu w elementy już istniejące w świecie wirtualnym, użytkownik żeby go zobaczyć musi fizycznie skierować głowę na dany element. Drugi z kolei wykorzystuje tradycyjny panel menu, ktory możemy spotkać na stronach internetowych czy w grach na komputer. Elementy interfejsu nie są częścią świata gry a sam panel jest nakładką na widok użytkownika. Kolejne podejście to interfejs przestrzenny, jest on połączeniem digetycznych i niedigetycznych.  Stanowi często część środowiska wirtualnego i jest w nim wyświtlany ale nie jest widoczny przez postacie. Ostatni typ to Meta, któy służy do reprezentacji statusu naszej postaci nie pojawiając się jednocześnie w świecie gry
%Interfejs diegetyczny – wpleciony w świat gry; użytkownik musi fizycznie spojrzeć na dany obiekt, aby zobaczyć informacje (np. zegarek na nadgarstku postaci).
%Interfejs niediegetyczny (klasyczny panel menu) – elementy UI nie są częścią świata gry, lecz nakładką na widok użytkownika (np. menu pauzy).
%Interfejs przestrzenny – zakotwiczony w przestrzeni wirtualnej, ale niekoniecznie istniejący w świecie gry z perspektywy postaci (np. menu przypięte do ściany).
%Interfejs statusowy (HUD) – reprezentuje stan postaci (np. poziom zdrowia), nie pojawiając się fizycznie w świecie gry.


\subsection{Opis typowych sposobów interakcji w VR}
typy interakcji z VR
Siedzący (Seated VR)najczęściej stosowany w symulatorach lotniczych i wyścigowych, które wymagają precyzyjnego sterowania; stojący zwany rownież stacjonarnym
Stacjonarny 
Swobodny ruch (Room-scale VR)
Tryby renderowania i wyświetlania VR
Tryby śledzenia ruchu użytkownika
Tryby użytkowania w zależności od platformy
Tryby użytkowania VR według zastosowania
Czym jest Interfejs uzytkownika w VR
Czym są interakcje i jakie wyroznia sie metody interakcji w vr

\subsection{Klasyczne zasady UX w kontekście środowisk VR}

Chociaż większość podstawowych zasad UX designu takich jak hierarchia wizualna, zasady Gestalt, spójność, czytelne affordance oraz informacje zwrotne są  uznawane za uniwersalane to projektowanie interfejsu użytkownika w oparciu o nie wymaga od projektanta innego podejścia. Wynika to z odmienności systemu VR od innych technologii, takich jak aplikacje mobilne, aplikacje desktopowe czy strony internetowe, w których interakcja z systemem odbywa się za pośrednictwem płaskiego ekranu oraz urządzeń wejścia, takich jak mysz, klawiatura czy ekran dotykowy.
W wirtualnej rzeczywistości interakcje odbywają sie w przestrzeni trójwymiarowej i wymagają aktywnosci ruchowej użytkownnika. Z tego powodu konieczne jest odpowiednie zastosowanie klasycznych zasad w sposób umożliwiający użytkownikom komfortową, intuicyjną oraz płynną interakcję z systemem.

Jedną z największych różnic w stosowaniu zasad UX w VR jest sposób nawigacji i poruszania się po środowisku wirtualnym. W tradycyjnych interfejsach użytkownik przemieszcza się po aplikacji za pomocą kliknięć, przewijania lub naciskania przycisków. W wirtualnej rzeczywistości nawigacja opiera się w dużej mierze na naturalnych ruchach użytkownika oraz mechanizmach lokomocji, takich jak teleportacja. Zmiana ta wpływa bezpośrednio na sposób projektowania hierarchii informacji oraz rozmieszczenia elementów interfejsu w przestrzeni.

Następna równie ważna różnica to \textbf{Interakcja w przestrzeni 3D}. W tradycyjnych interfejsach użytkownik wchodzi w interakcje z płaskimi, dwuwymiarowymi elementami na ekranie. W VR interakcje odbywają się w sposób bezpośredni i możliwe jest manipulowanie obiektami w sposób zbliżony do rzeczywistych czynności, takich jak chwytanie, obracanie czy przesuwanie elementów tak jakby były on prawdziwe. Taki sposób interakcji wzmacnia immersję, lecz jednocześnie wymaga szczególnej dbałości o czytelne affordance oraz natychmiastową informację zwrotną, aby użytkownik mógł poprawnie interpretować skutki swoich działań.

Adaptacja klasycznych zasad UX do środowiska VR wiąże się również z koniecznością unikania określonych rozwiązań projektowych, które mogą negatywnie wpływać na komfort i odbiór doświadczenia. Nadmierna liczba elementów interfejsu prowadzi do przeciążenia poznawczego i zaburza immersję, dlatego projektowanie powinno opierać się na minimalizmie oraz koncentracji na kluczowych funkcjach, przy jednoczesnym ukrywaniu lub eliminowaniu elementów drugorzędnych. Równie istotne jest unikanie nagłych zmian perspektywy i gwałtownych ruchów kamery, które mogą powodować dezorientację, zawroty głowy, a w skrajnych przypadkach także mdłości. Ruch w środowisku wirtualnym powinien być płynny i przewidywalny, naśladując naturalne ruchy głowy użytkownika oraz wspierany przez stopniowe przejścia i animacje. Ponadto brak spójnych i powtarzalnych schematów interakcji może prowadzić do frustracji użytkowników; stosowanie ujednoliconych rozwiązań tam, gdzie jest to możliwe, ułatwia adaptację do środowiska VR, skraca czas nauki obsługi systemu oraz pozytywnie wpływa na ogólne doświadczenie użytkownika (Swink, Game Feel).


\section{Tło technologiczne VR} 

Kluczowym etapem projektowania tej pracy był wybór odpowiednich narzędzi umożliwiających prototypownie, implementację oraz testowanie interfejsu w środowisku VR. Wybór używanej technologii jest kluczowy, aby zapewnić wysoką jakość doświadczeń użytkownika oraz zachować efektywność samego procesu projektowania.


\subsection{Silniki gier i środowiska VR (Unity, Unreal, Godot)}

\textit{Unity} to jeden z najpopularniejszych wieloplatformowych silników do tworzenia gier wideo oraz interaktywnych aplikacji. Umożliwia zaawansowaną obsługę grafiki i fizyki, a w samym edytorze udostępniono rozbudowany zestaw narzędzi do projektowania scen, animacji i efektów graficznych \ref{unity_engine_example}. Proces programowania odbywa się przy użyciu języka \textit{C\#}.

W Unity dodano również wsparcie dla wirtualnej oraz rozszerzonej rzeczywistości, co przekłada się na szerokie zastosowanie tej technologii w szkoleniach, symulacjach czy działaniach marketingowych. Oprócz możliwości tworzenia zaawansowanych pod względem wizualnym scen, istotna jest też opcja łatwej integracji gotowych wtyczek obsługujących urządzenia AR i VR od różnych producentów. Rozbudowany system oświetlenia oraz post-processingu zapewnia szeroki wachlarz opcji i umożliwia dostosowanie grafiki pod dedykowane urządzenia, dzięki czemu można uzyskać zadowalającą jakość i odpowiednią optymalizację na urządzenia mobilne oraz wysoce realistyczną grafikę na komputerach osobistych.

\begin{figure}[!htb]
    \centering
    \includegraphics[width=0.8\textwidth]{images/unity.png}
    \caption{Przykładowy wygląd edytora Unity}
    \label{unity_engine_example}
\end{figure}

Regularne aktualizacje silnika rozszerzają jego funkcjonalność i dodają nowe rozwiązania, takie jak obsługa ray tracingu, DLSS oraz nowe narzędzia. Stały rozwój sprawia, że Unity zachowuje elastyczność w obliczu zmieniających się potrzeb rynku, a jego wszechstronność umożliwia realizację nawet najbardziej rozbudowanych koncepcji interaktywnych.

Społeczność skupiona wokół tego silnika udostępnia liczne materiały edukacyjne w postaci kursów i filmów, co znacząco obniża poziom wejścia, przyspiesza naukę oraz ułatwia rozwiązywanie ewentualnych problemów technicznych. Dokumentacja silnika jest regularnie rozwijana, a oficjalny sklep \textit{Asset Store} zapewnia dostęp do bardzo dużej ilości bezpłatnych oraz płatnych pakietów, obejmujących zasoby takie jak modele, tekstury i dźwięki oraz dodatkowe użyteczne narzędzia wraz z całymi gotowymi systemami ułatwiającymi stworzenie projektu i pozwalającymi ograniczyć koszty produkcji.


\subsection{Biblioteki i frameworki VR (XR Toolkit, OpenXR, SteamVR)}

\subsection{Narzędzia do projektowania interfejsów (Figma, Blender)}

Figma to podstawowe narzędzie służące do projektowania interfejsów użytkownika, jednocześnie umożliwia tworzenie interaktywnych klikalnych prototypów aplikacji. Jest to aktualnie jedno z najbardziej popularnych i bezpłatnych narzędzi w branży UX/UI. W projekcie Figma zostanie wykorzystana do stworzenia wstępnych projektów i prototypów interfejsów, a następnie do przetestowania rożnych układów elementów interfejsu. 

\subsection{Narzędzia do analizy UX/UI (Google Forms, UEQ, SUS)}

Gogle VR to urządzenia, które całkowicie zasłaniają pole widzenia i wyświetlają przez użytkownikiem dwa identyczne, ale nieco przesunięte od siebie obrazy, które nałożone na siebie przez mózg dają wrażenie głębi. Wewnątrz gogli umieszczone są specjalne czujniki takie jak żyroskop, kamery i inne śledzące położenie i ruch głowy w przestrzeni. Dzięki temu użytkownik obracając głową faktycznie rozgląda się w przestrzeni wirtualnej obracając kamerą gracza.

Oprócz gogli istotne są również specjalne kontrolery zakładane na dłonie. Dzięki nim możliwa jest interakcja z wirtualnym otoczeniem, a specjalne czujniki i przyciski wykrywają pozycje palców i pozwalają określić, czy gracz w tej chwili próbuje złapać przedmiot. Całość wraz z systemem śledzenia pozycji dłoni pozwala dowolnie sięgać w różnych kierunkach, co pozwala mieć wpływ na obiekty w aplikacji. Gracz może na przykład chwycić przedmiot i rzucić nim, co pozwoli wywołać kolejne symulacje w fizyce.

Oprócz tego istnieje wiele różnych dodatków rozszerzających możliwości wirtualnej rzeczywistości, jak specjalne kombinezony monitorujących ruch oraz umożliwiających odczuwanie na skórze poprzez elektrostymulację nerwów i mięśni lub kapsuły, w których użytkownik ma możliwość skakania oraz poruszania się w dowolny sposób bez ryzyka, że uszkodzi coś w pokoju.

2.9. Sprzęt i konfiguracja testowa


\chapter{Analiza istniejących rozwiązań}

\section{ Cel i zakres analizy}
Opis celu: identyfikacja sposobów realizacji kluczowych funkcji UI w grach VR.
Wymienienie analizowanych funkcji: menu pauzy, ekwipunek/inwentarz, tryb poruszania się, informacja zwrotna.
Wyjaśnienie, że analiza służy jako podstawa do dalszej weryfikacji ankietowej


\section{Metoda analizy: heurystyczna ocena wzorców projektowych}
Opis mojego dwuetapowego  podejścia:

Identyfikacja wzorców jak to się robi?
Ocena według heurystyk NN/g (2021) dostosowanych do VR.
– Lista zastosowanych heurystyk (bez szczegółów naruszeń):
H1. Visibility of System Status
H2. Match Between System and the Real World
H3. User Control and Freedom
H4. Consistency and Standards
H5. Error Prevention


\section{ Materiał badawczy (wybór gier VR)}
– Kryteria wyboru:

popularność
różnorodność gatunków
dostępność na komercyjnych platformach
obecność interfejsu w przestrzeni 3D

Lista gier:

\section{Identyfikacja wzorców projektowych}
\section{Ustalenie zestawu heurystyk do oceny}
\section{Ocena gier według heurystyk}
\section{Podsumowanie analizy i Wnioski}


\chapter{Projekt własnego interfejsu menu w VR}

\chapter{Prototypowanie i Implementacja}
\section{Prototypowanie}
\subsection{Proces projektowy UX (Opis projektowania UX}
%\subsection{ Personas i user flows (opcjonalnie)
\subsection{ Projektowanie interfejsu zgodnie z zasadami UX}
\subsection{ Makiety low-fidelity}
\subsection{ Prototypy high-fidelity}
\subsection{ Testy prototypów (first-click, A/B, tree testing)}
\subsection{ Wnioski projektowe do VR}


\section{Implementacja}

\subsection{ Środowisko i konfiguracja projektu Unity VR}
\subsection{  Architektura interfejsu VR}
\subsection{ Implementacja elementów UI (2D, 3D)}
\subsection{ Implementacja interakcji (pointer, direct touch, gesty)}
\subsection{ Wdrażanie rozwiązań UX z prototypu Figma}
\subsection{ Testy techniczne, optymalizacja i stabilność}
%\subsection{Podsumowanie implementacji}


\summary
Założony cel tej pracy, aby 


\bibliographystyle{plain}
\printbibliography[type=book,title={Books only}]

% spis rysunków (jeżeli jest potrzebny):
\listoffigures

% spis listingów (jeżeli jest potrzebny):
\lstlistoflistings

\end{document}
